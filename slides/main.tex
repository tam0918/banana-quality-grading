% ==============================================================================
% SLIDE BÁO CÁO: ỨNG DỤNG THỊ GIÁC MÁY TÍNH TRONG PHÂN LOẠI CHẤT LƯỢNG CHUỐI
% ==============================================================================
\documentclass[aspectratio=169,11pt]{beamer}

% === PACKAGES ===
\usepackage[utf8]{inputenc}
\usepackage[vietnamese]{babel}
\usepackage{graphicx}
\usepackage{booktabs}
\usepackage{amsmath}
\usepackage{tikz}
\usepackage{pgfplots}
\usepackage{subcaption}
\usepackage{fontawesome5}
\usepackage{setspace}

% === THEME ===
\usetheme{Madrid}
\usecolortheme{whale}
\setbeamertemplate{navigation symbols}{}
\setbeamertemplate{footline}{
    \leavevmode%
    \hbox{%
        \begin{beamercolorbox}[wd=.30\paperwidth,ht=2.25ex,dp=1ex,center]{author in head/foot}%
            \usebeamerfont{author in head/foot}\insertshortauthor
        \end{beamercolorbox}%
        \begin{beamercolorbox}[wd=.40\paperwidth,ht=2.25ex,dp=1ex,center]{title in head/foot}%
            \usebeamerfont{title in head/foot}\insertshorttitle
        \end{beamercolorbox}%
        \begin{beamercolorbox}[wd=.30\paperwidth,ht=2.25ex,dp=1ex,right]{date in head/foot}%
            \usebeamerfont{date in head/foot}%
            \raisebox{-0.3ex}{\includegraphics[height=1.8ex]{images/hus_logo.png}}\hspace{1ex}%
            \insertframenumber{} / \inserttotalframenumber\hspace*{2ex}
        \end{beamercolorbox}}%
    \vskip0pt%
}

% === COLORS ===
\definecolor{bananaYellow}{RGB}{255, 225, 53}
\definecolor{bananaGreen}{RGB}{138, 180, 71}
\definecolor{bananaBrown}{RGB}{139, 90, 43}
\definecolor{primaryBlue}{RGB}{0, 82, 147}
\definecolor{husBlue}{RGB}{0, 51, 102}

\setbeamercolor{title}{fg=white,bg=husBlue}
\setbeamercolor{frametitle}{fg=white,bg=husBlue}
\setbeamercolor{block title}{fg=white,bg=husBlue}
\setbeamercolor{structure}{fg=husBlue}
\setbeamercolor{author in head/foot}{fg=white,bg=husBlue!80}
\setbeamercolor{title in head/foot}{fg=white,bg=husBlue!60}
\setbeamercolor{date in head/foot}{fg=white,bg=husBlue!40}

% === SPACING ===
\setbeamersize{text margin left=8mm,text margin right=8mm}

% === TITLE INFO ===
\title[Phân loại chất lượng Chuối]{Ứng dụng Thị giác Máy tính trong\\Phân loại Chất lượng Chuối}
\subtitle{Banana Quality Grading using Computer Vision}
\author[Lường Văn Tâm, Khương Thanh Tín]{%
    \textbf{Sinh viên thực hiện:}\\
    Lường Văn Tâm -- Khương Thanh Tín\\[0.3cm]
    \textbf{Giảng viên hướng dẫn:}\\
    PGS. TS. Phạm Tiến Lâm
}
\institute[HUS-VNU]{
    Khoa Vật lý\\
    Trường Đại học Khoa học Tự nhiên\\
    Đại học Quốc gia Hà Nội
}
\date{Tháng 01/2026}

% === LOGO ===
\titlegraphic{
    \includegraphics[height=1.5cm]{images/hus_logo.png}
}
% Logo đã được đặt trong footer, không cần \logo{} riêng

\begin{document}

% ==============================================================================
% TITLE SLIDE
% ==============================================================================
\begin{frame}[plain]
    \vspace{0.5cm}
    \titlepage
\end{frame}

% ==============================================================================
% OUTLINE
% ==============================================================================
\begin{frame}{Nội dung trình bày}
    \tableofcontents[hideallsubsections]
\end{frame}

% ==============================================================================
% SECTION 1: GIỚI THIỆU
% ==============================================================================
\section{Giới thiệu}

\begin{frame}{Đặt vấn đề}
    \begin{columns}[T]
        \column{0.58\textwidth}
        \textbf{\large Thách thức trong ngành chuối:}
        \vspace{0.3cm}
        \begin{itemize}
            \setlength{\itemsep}{6pt}
            \item Phân loại thủ công tốn nhân lực và thời gian
            \item Thiếu tính nhất quán trong đánh giá chất lượng
            \item Khó kiểm soát chất lượng ở quy mô lớn
            \item Chuối là trái cây xuất khẩu quan trọng của Việt Nam
        \end{itemize}
        
        \column{0.38\textwidth}
        \centering
        \includegraphics[width=0.95\textwidth]{images/train_batch0.jpg}
        
        \vspace{0.2cm}
        {\small\textit{Các loại chuối với độ chín khác nhau}}
    \end{columns}
\end{frame}

\begin{frame}{Giải pháp đề xuất}
    \begin{block}{Ý tưởng chính}
        Ứng dụng \textbf{AI \& Computer Vision} để phân loại tự động, real-time
    \end{block}
    
    \vspace{0.5cm}
    
    \begin{columns}[T]
        \column{0.32\textwidth}
        \centering
        \textcolor{husBlue}{\faRobot}\\[0.3cm]
        \textbf{Tự động hóa}\\[0.2cm]
        {\small Giảm chi phí nhân công, tăng năng suất}
        
        \column{0.32\textwidth}
        \centering
        \textcolor{husBlue}{\faClock}\\[0.3cm]
        \textbf{Real-time}\\[0.2cm]
        {\small Xử lý video webcam trực tiếp}
        
        \column{0.32\textwidth}
        \centering
        \textcolor{husBlue}{\faCheckCircle}\\[0.3cm]
        \textbf{Nhất quán}\\[0.2cm]
        {\small Đánh giá khách quan, đồng nhất}
    \end{columns}
\end{frame}

\begin{frame}{Mục tiêu nghiên cứu}
    \begin{block}{Mục tiêu chính}
        Xây dựng hệ thống phân loại chất lượng chuối \textbf{real-time} sử dụng YOLOv8
    \end{block}
    
    \vspace{0.5cm}
    
    \textbf{\large Phân loại 4 cấp độ chất lượng:}
    
    \vspace{0.4cm}
    
    \begin{columns}[T]
        \column{0.24\textwidth}
        \centering
        \textcolor{bananaGreen}{\Large\faLeaf}\\[0.3cm]
        \textbf{Unripe}\\[0.2cm]
        Chuối xanh\\
        {\small Chưa thu hoạch}
        
        \column{0.24\textwidth}
        \centering
        \textcolor{bananaYellow}{\Large\faStar}\\[0.3cm]
        \textbf{Export}\\[0.2cm]
        Chín vừa\\
        {\small Xuất khẩu}
        
        \column{0.24\textwidth}
        \centering
        \textcolor{bananaBrown}{\Large\faExclamationTriangle}\\[0.3cm]
        \textbf{Overripe}\\[0.2cm]
        Quá chín\\
        {\small Bán gấp}
        
        \column{0.24\textwidth}
        \centering
        \textcolor{red}{\Large\faTimesCircle}\\[0.3cm]
        \textbf{Defective}\\[0.2cm]
        Hỏng/Thối\\
        {\small Loại bỏ}
    \end{columns}
\end{frame}

% ==============================================================================
% SECTION 2: PHƯƠNG PHÁP
% ==============================================================================
\section{Phương pháp}

\begin{frame}{Kiến trúc hệ thống 2-Stage Pipeline}
    \centering
    \begin{tikzpicture}[scale=0.75, transform shape,
        block/.style={rectangle, draw, fill=blue!15, text width=2cm, text centered, rounded corners, minimum height=0.9cm, font=\small},
        arrow/.style={->, thick, >=stealth}
    ]
        % Input
        \node[block, fill=green!25] (input) at (0,0) {Video\\Frame};
        
        % Detector
        \node[block, fill=blue!20] (detector) at (3,0) {YOLO\\Detector};
        
        % Classifier
        \node[block, fill=orange!25] (classifier) at (6,0) {YOLO\\Classifier};
        
        % Analyzer
        \node[block, fill=purple!20] (analyzer) at (9,0) {Banana\\Analyzer};
        
        % Output
        \node[block, fill=red!25] (output) at (12,0) {Kết quả\\Phân loại};
        
        % Arrows
        \draw[arrow] (input) -- (detector);
        \draw[arrow] (detector) -- node[above, font=\scriptsize] {crop} (classifier);
        \draw[arrow] (classifier) -- (analyzer);
        \draw[arrow] (analyzer) -- (output);
    \end{tikzpicture}
    
    \vspace{0.6cm}
    
    \begin{columns}[T]
        \column{0.48\textwidth}
        \begin{block}{Stage 1: Detection}
            \begin{itemize}
                \setlength{\itemsep}{3pt}
                \item COCO pretrained YOLOv8n
                \item Định vị chuối trong frame
                \item Không cần annotation bbox
            \end{itemize}
        \end{block}
        
        \column{0.48\textwidth}
        \begin{block}{Stage 2: Classification}
            \begin{itemize}
                \setlength{\itemsep}{3pt}
                \item YOLOv8n-cls fine-tuned
                \item Phân loại 4 class độ chín
                \item + BananaAnalyzer refinement
            \end{itemize}
        \end{block}
    \end{columns}
\end{frame}

\begin{frame}{Tại sao chọn 2-Stage Pipeline?}
    \begin{alertblock}{Constraint → Opportunity}
        Dataset Kaggle chỉ có \textbf{classification labels}, không có bounding box annotation!
    \end{alertblock}
    
    \vspace{0.4cm}
    
    \begin{columns}[T]
        \column{0.48\textwidth}
        \textbf{\large Giải pháp thông minh:}
        \vspace{0.3cm}
        \begin{enumerate}
            \setlength{\itemsep}{8pt}
            \item Dùng COCO detector có sẵn (class ``banana'')
            \item Fine-tune classifier riêng trên Kaggle data
            \item Kết hợp với BananaAnalyzer cho edge cases
        \end{enumerate}
        
        \column{0.48\textwidth}
        \textbf{\large Lợi ích:}
        \vspace{0.3cm}
        \begin{itemize}
            \setlength{\itemsep}{8pt}
            \item[\textcolor{green}{\faCheck}] Không cần label bbox thủ công
            \item[\textcolor{green}{\faCheck}] Tận dụng pretrained knowledge
            \item[\textcolor{green}{\faCheck}] Cải tiến từng component độc lập
            \item[\textcolor{green}{\faCheck}] Flexibility cao
        \end{itemize}
    \end{columns}
\end{frame}

\begin{frame}{Module BananaAnalyzer -- Trích xuất đặc trưng}
    \begin{columns}[T]
        \column{0.48\textwidth}
        \textbf{\large Đặc trưng màu sắc (HSV/LAB):}
        \vspace{0.3cm}
        \begin{table}[h]
            \small
            \begin{tabular}{ll}
                \toprule
                \textbf{Feature} & \textbf{Ý nghĩa} \\
                \midrule
                yellow\_ratio & Tỉ lệ màu vàng (chín) \\
                green\_ratio & Tỉ lệ màu xanh (chưa chín) \\
                brown\_ratio & Tỉ lệ màu nâu (quá chín) \\
                black\_ratio & Tỉ lệ màu đen (hỏng) \\
                spot\_count & Số lượng đốm đen \\
                quality\_score & Điểm chất lượng tổng hợp \\
                \bottomrule
            \end{tabular}
        \end{table}
        
        \column{0.48\textwidth}
        \textbf{\large Vai trò:}
        \vspace{0.3cm}
        \begin{itemize}
            \setlength{\itemsep}{6pt}
            \item Bổ sung thông tin cho classifier
            \item Kết hợp domain knowledge với deep learning
            \item Cung cấp explainable features
        \end{itemize}
    \end{columns}
\end{frame}

\begin{frame}{Module BananaAnalyzer -- Feature Refinement}
    \begin{block}{Khi nào kích hoạt refinement?}
        \begin{itemize}
            \item Khi classifier không chắc chắn (confidence < 50\%)
            \item Khi phát hiện dấu hiệu defective rõ ràng
        \end{itemize}
    \end{block}
    
    \vspace{0.4cm}
    
    \textbf{\large Quy tắc phát hiện Defective:}
    \vspace{0.3cm}
    
    \begin{columns}[T]
        \column{0.48\textwidth}
        \begin{exampleblock}{Điều kiện 1}
            \texttt{black\_ratio > 0.15}
            
            \vspace{0.2cm}
            {\small → Vùng đen chiếm >15\% diện tích}
        \end{exampleblock}
        
        \column{0.48\textwidth}
        \begin{exampleblock}{Điều kiện 2}
            \texttt{brown\_ratio > 0.3 AND spot\_count > 10}
            
            \vspace{0.2cm}
            {\small → Nhiều đốm nâu + số đốm cao}
        \end{exampleblock}
    \end{columns}
    
    \vspace{0.4cm}
    \centering
    {\small → Sửa khoảng \textbf{10-15\%} edge cases so với chỉ dùng classifier}
\end{frame}

% ==============================================================================
% SECTION 3: DỮ LIỆU VÀ HUẤN LUYỆN
% ==============================================================================
\section{Dữ liệu \& Huấn luyện}

\begin{frame}{Dataset: Kaggle Banana Classification}
    \begin{columns}[T]
        \column{0.45\textwidth}
        \textbf{\large Thông tin dataset:}
        \vspace{0.3cm}
        \begin{table}[h]
            \small
            \begin{tabular}{lr}
                \toprule
                \textbf{Thuộc tính} & \textbf{Giá trị} \\
                \midrule
                Tổng số ảnh & $\sim$13,500 \\
                Số class & 4 \\
                Dung lượng & 227 MB \\
                License & MIT \\
                \bottomrule
            \end{tabular}
        \end{table}
        
        \vspace{0.3cm}
        \textbf{\large Phân chia dữ liệu:}
        \begin{itemize}
            \item Training: 80\%
            \item Validation: 10\%
            \item Test: 10\%
        \end{itemize}
        
        \column{0.50\textwidth}
        \centering
        \includegraphics[width=0.9\textwidth]{images/train_batch1.jpg}
        
        \vspace{0.2cm}
        {\small\textit{Ví dụ ảnh từ training set}}
    \end{columns}
\end{frame}

\begin{frame}{4 Classes phân loại}
    \begin{columns}[T]
        \column{0.24\textwidth}
        \centering
        \textcolor{bananaGreen}{\Large\faCircle}\\[0.3cm]
        \textbf{1. Unripe}\\[0.2cm]
        {\small Chuối xanh, chưa chín}
        
        \column{0.24\textwidth}
        \centering
        \textcolor{bananaYellow}{\Large\faCircle}\\[0.3cm]
        \textbf{2. Ripe}\\[0.2cm]
        {\small Chín vàng, sẵn sàng ăn}
        
        \column{0.24\textwidth}
        \centering
        \textcolor{bananaBrown}{\Large\faCircle}\\[0.3cm]
        \textbf{3. Overripe}\\[0.2cm]
        {\small Quá chín, có đốm nâu}
        
        \column{0.24\textwidth}
        \centering
        \textcolor{black}{\Large\faCircle}\\[0.3cm]
        \textbf{4. Rotten}\\[0.2cm]
        {\small Thối, hỏng hoàn toàn}
    \end{columns}
    
    \vspace{0.8cm}
    
    \begin{alertblock}{Lưu ý về mapping}
        Trong hệ thống thực tế: \textbf{Ripe → Export}, \textbf{Rotten → Defective}
        
        (Phù hợp với ngữ cảnh phân loại thương mại)
    \end{alertblock}
\end{frame}

\begin{frame}{Quá trình huấn luyện}
    \begin{columns}[T]
        \column{0.45\textwidth}
        \textbf{\large Cấu hình training:}
        \vspace{0.3cm}
        \begin{table}[h]
            \small
            \begin{tabular}{ll}
                \toprule
                \textbf{Tham số} & \textbf{Giá trị} \\
                \midrule
                Base model & yolov8n-cls.pt \\
                Image size & 416×416 \\
                Epochs & 50 \\
                Batch size & Auto \\
                Optimizer & AdamW \\
                Device & GPU (CUDA) \\
                \bottomrule
            \end{tabular}
        \end{table}
        
        \vspace{0.3cm}
        \textbf{Thời gian training:} $\sim$71 phút
        
        \column{0.50\textwidth}
        \centering
        \includegraphics[width=\textwidth]{images/results.png}
        
        \vspace{0.2cm}
        {\small\textit{Training metrics qua 50 epochs}}
    \end{columns}
\end{frame}

% ==============================================================================
% SECTION 4: KẾT QUẢ
% ==============================================================================
\section{Kết quả}

\begin{frame}{Kết quả Classifier trên Validation Set}
    \begin{columns}[T]
        \column{0.42\textwidth}
        \begin{block}{Metrics chính}
            \begin{itemize}
                \setlength{\itemsep}{8pt}
                \item \textbf{Top-1 Accuracy: 98.75\%}
                \item Top-5 Accuracy: 100\%
                \item Training Loss: 0.0096
                \item Validation Loss: 0.0752
            \end{itemize}
        \end{block}
        
        \vspace{0.4cm}
        \textbf{Nhận định:}
        \begin{itemize}
            \setlength{\itemsep}{4pt}
            \item Model hội tụ nhanh ($\sim$15 epochs)
            \item Transfer learning hiệu quả
        \end{itemize}
        
        \column{0.54\textwidth}
        \centering
        \includegraphics[width=0.95\textwidth]{images/confusion_matrix_normalized.png}
        
        \vspace{0.2cm}
        {\small\textit{Confusion Matrix (normalized)}}
    \end{columns}
\end{frame}

\begin{frame}{Phân tích Confusion Matrix}
    \begin{columns}[T]
        \column{0.52\textwidth}
        \centering
        \includegraphics[width=0.95\textwidth]{images/confusion_matrix.png}
        
        \vspace{0.2cm}
        {\small\textit{Confusion Matrix (raw counts)}}
        
        \column{0.44\textwidth}
        \textbf{\large Nhận định quan trọng:}
        \vspace{0.4cm}
        \begin{itemize}
            \setlength{\itemsep}{10pt}
            \item Model học được \textbf{thứ tự chín}
            \item Lỗi chỉ xảy ra giữa class \textbf{liền kề}:
            \begin{itemize}
                \item ripe $\leftrightarrow$ overripe
                \item overripe $\leftrightarrow$ rotten
            \end{itemize}
            \item \textbf{Không có} nhầm lẫn cực đoan\\(unripe $\leftrightarrow$ rotten)
        \end{itemize}
        
        \vspace{0.4cm}
        \begin{block}{Kết luận}
            Feature space phù hợp với bản chất tự nhiên!
        \end{block}
    \end{columns}
\end{frame}

\begin{frame}{Pipeline hoàn chỉnh -- Test trên ảnh thực tế}
    \begin{columns}[T]
        \column{0.45\textwidth}
        \textbf{\large Test set: 15 ảnh thực tế}
        \vspace{0.3cm}
        \begin{table}[h]
            \small
            \begin{tabular}{lcc}
                \toprule
                \textbf{Category} & \textbf{Frame} & \textbf{Instance} \\
                \midrule
                Export & 4 & 7 \\
                Unripe & --- & 9 \\
                Overripe & 3 & 3 \\
                Defective & 4 & 4 \\
                None (negative) & 4 & --- \\
                \bottomrule
            \end{tabular}
        \end{table}
        
        \vspace{0.3cm}
        \textbf{Kết quả:}
        \begin{itemize}
            \item[\textcolor{green}{\faCheck}] Negative samples xử lý đúng
            \item[\textcolor{green}{\faCheck}] Multi-object detection OK
            \item[\textcolor{green}{\faCheck}] Feature refinement hoạt động
        \end{itemize}
        
        \column{0.50\textwidth}
        \centering
        \includegraphics[width=0.95\textwidth]{images/real_e2e.jpg}
        
        \vspace{0.2cm}
        {\small\textit{Kết quả test end-to-end}}
    \end{columns}
\end{frame}

\begin{frame}{Ví dụ kết quả Pipeline (1/2)}
    \vspace{0.3cm}
    \begin{columns}[T]
        \column{0.32\textwidth}
        \centering
        \includegraphics[width=0.95\textwidth]{images/pipeline_results/3.jpg}
        
        \vspace{0.2cm}
        {\small \textbf{Overripe} (100\%)}
        
        \column{0.32\textwidth}
        \centering
        \includegraphics[width=0.95\textwidth]{images/pipeline_results/5.jpg}
        
        \vspace{0.2cm}
        {\small \textbf{Export} (100\%)}
        
        \column{0.32\textwidth}
        \centering
        \includegraphics[width=0.95\textwidth]{images/pipeline_results/7.jpg}
        
        \vspace{0.2cm}
        {\small \textbf{Defective} (99.96\%)}
    \end{columns}
    
    \vspace{0.5cm}
    \centering
    {\small → Confidence cao cho các trường hợp rõ ràng}
\end{frame}

\begin{frame}{Ví dụ kết quả Pipeline (2/2) -- Multi-object}
    \vspace{0.3cm}
    \begin{columns}[T]
        \column{0.48\textwidth}
        \centering
        \includegraphics[width=0.85\textwidth]{images/pipeline_results/14.jpg}
        
        \vspace{0.3cm}
        {\small \textbf{Nải 6 quả:} 4 Unripe + 2 Export}
        
        \vspace{0.2cm}
        {\footnotesize → Phát hiện từng quả riêng biệt}
        
        \column{0.48\textwidth}
        \centering
        \includegraphics[width=0.85\textwidth]{images/pipeline_results/15.jpg}
        
        \vspace{0.3cm}
        {\small \textbf{Nải 6 quả:} Overall = Defective}
        
        \vspace{0.2cm}
        {\footnotesize → Aggregation theo worst case}
    \end{columns}
\end{frame}

\begin{frame}{Hiệu năng Real-time}
    \begin{columns}[T]
        \column{0.48\textwidth}
        \textbf{\large Benchmark trên các hardware:}
        \vspace{0.3cm}
        \begin{table}[h]
            \small
            \begin{tabular}{lcc}
                \toprule
                \textbf{Hardware} & \textbf{Latency} & \textbf{FPS} \\
                \midrule
                RTX 3060 & $\sim$13 ms & 40-75 \\
                GTX 1650 & $\sim$23 ms & 30-40 \\
                Intel i7 (CPU) & $\sim$33 ms & 25-30 \\
                Intel i5 (CPU) & $\sim$40 ms & 25-30 \\
                \bottomrule
            \end{tabular}
        \end{table}
        
        \vspace{0.3cm}
        \textbf{Kết luận:}
        \begin{itemize}
            \item GPU mid-range: \textbf{Real-time mượt}
            \item CPU: \textbf{Real-time tốt} ($\sim$30 FPS)
        \end{itemize}
        
        \column{0.48\textwidth}
        \begin{block}{Công thức Latency tổng}
            \small
            \vspace{0.2cm}
            $\text{Latency} = T_{\text{capture}} + T_{\text{det}} + T_{\text{cls}}$
            
            $\quad\quad\quad\quad + T_{\text{analyzer}} + T_{\text{render}}$
            \vspace{0.2cm}
        \end{block}
        
        \vspace{0.4cm}
        \begin{exampleblock}{Trên GPU mid-range}
            \centering
            $\text{Latency}_{\text{total}} \approx 50-80$ ms
            
            \vspace{0.2cm}
            {\small → Đủ cho ứng dụng thực tế}
        \end{exampleblock}
    \end{columns}
\end{frame}

\begin{frame}{So sánh: Classifier ``Thô'' vs Pipeline Hoàn chỉnh}
    \vspace{0.3cm}
    \begin{table}[h]
        \begin{tabular}{lcc}
            \toprule
            \textbf{Tiêu chí} & \textbf{Giai đoạn 1} & \textbf{Giai đoạn 2} \\
            \midrule
            Dữ liệu đánh giá & Kaggle val set & Ảnh thực tế \\
            Có localization & \textcolor{red}{\faTimes} & \textcolor{green}{\faCheck} \\
            Xử lý nhiều quả & \textcolor{red}{\faTimes} & \textcolor{green}{\faCheck} \\
            Feature refinement & \textcolor{red}{\faTimes} & \textcolor{green}{\faCheck} \\
            Edge case handling & \textcolor{red}{\faTimes} & \textcolor{green}{\faCheck} \\
            Accuracy báo cáo & 98.75\% (clean) & Context-dependent \\
            \bottomrule
        \end{tabular}
    \end{table}
    
    \vspace{0.5cm}
    
    \begin{alertblock}{Bài học quan trọng}
        \centering
        \textbf{Accuracy trên validation set $\neq$ Hiệu năng thực tế!}
        
        {\small Cần đánh giá end-to-end trên dữ liệu thực}
    \end{alertblock}
\end{frame}

% ==============================================================================
% SECTION 5: DEMO & KẾT LUẬN
% ==============================================================================
\section{Demo \& Kết luận}

\begin{frame}{Demo hệ thống}
    \centering
    \vspace{0.3cm}
    \begin{tikzpicture}
        \node[draw, rounded corners, fill=husBlue!10, minimum width=11cm, minimum height=4.5cm] (demo) {};
        \node at (demo.center) {\Large\textbf{DEMO VIDEO / LIVE DEMO}};
    \end{tikzpicture}
    
    \vspace{0.5cm}
    
    \begin{columns}[T]
        \column{0.48\textwidth}
        \textbf{Tính năng chính:}
        \begin{itemize}
            \setlength{\itemsep}{4pt}
            \item Real-time webcam processing
            \item Bounding box + label tiếng Việt
            \item Hiển thị confidence + quality score
        \end{itemize}
        
        \column{0.48\textwidth}
        \textbf{Cách chạy:}
        \begin{itemize}
            \setlength{\itemsep}{4pt}
            \item Python 3.9+ với PyTorch
            \item Webcam (hoặc video file)
            \item GPU khuyến nghị
        \end{itemize}
        
        \vspace{0.3cm}
        {\small\texttt{python main.py}}
    \end{columns}
\end{frame}

\begin{frame}{Đóng góp chính của đề tài}
    \begin{enumerate}
        \setlength{\itemsep}{12pt}
        \item \textbf{Kiến trúc 2-stage pipeline hiệu quả:}
        \begin{itemize}
            \item Giải quyết vấn đề thiếu bbox annotation
            \item Tận dụng COCO pretrained + Kaggle classification data
        \end{itemize}
        
        \item \textbf{Module BananaAnalyzer sáng tạo:}
        \begin{itemize}
            \item Kết hợp deep learning + domain knowledge (HSV/LAB)
            \item Feature refinement cho edge cases
        \end{itemize}
        
        \item \textbf{Hệ thống real-time hoàn chỉnh:}
        \begin{itemize}
            \item 98.75\% accuracy trên validation
            \item 40-75 FPS trên GPU mid-range
            \item Giao diện tiếng Việt thân thiện
        \end{itemize}
    \end{enumerate}
\end{frame}

\begin{frame}{Hạn chế của nghiên cứu}
    \begin{columns}[T]
        \column{0.48\textwidth}
        \textbf{\large Hạn chế kỹ thuật:}
        \vspace{0.3cm}
        \begin{itemize}
            \setlength{\itemsep}{8pt}
            \item Phụ thuộc COCO detector (class ``banana'')
            \item Dataset nền đơn giản, ít nhiễu
            \item Ranh giới mờ giữa overripe/rotten
        \end{itemize}
        
        \column{0.48\textwidth}
        \textbf{\large Hạn chế về phạm vi:}
        \vspace{0.3cm}
        \begin{itemize}
            \setlength{\itemsep}{8pt}
            \item Chưa test nhiều giống chuối khác nhau
            \item Chưa đánh giá trên production scale
            \item Thiếu user study định lượng
        \end{itemize}
    \end{columns}
\end{frame}

\begin{frame}{Hướng phát triển}
    \begin{columns}[T]
        \column{0.48\textwidth}
        \textbf{\large Ngắn hạn:}
        \vspace{0.3cm}
        \begin{itemize}
            \setlength{\itemsep}{8pt}
            \item Train detector riêng cho banana
            \item Export TFLite/ONNX cho mobile
            \item Cải thiện UI/UX
        \end{itemize}
        
        \column{0.48\textwidth}
        \textbf{\large Dài hạn:}
        \vspace{0.3cm}
        \begin{itemize}
            \setlength{\itemsep}{8pt}
            \item Mở rộng sang trái cây khác
            \item Dự đoán shelf-life (hạn sử dụng)
            \item Explainable AI (Grad-CAM)
            \item Deploy production system
        \end{itemize}
    \end{columns}
\end{frame}

\begin{frame}{Kết luận}
    \begin{block}{Tổng kết}
        Đề tài đã xây dựng thành công hệ thống phân loại chất lượng chuối real-time:
        \vspace{0.3cm}
        \begin{itemize}
            \setlength{\itemsep}{6pt}
            \item \textbf{98.75\% Top-1 Accuracy} trên validation set
            \item \textbf{40-75 FPS} trên GPU mid-range
            \item Pipeline 2-stage linh hoạt, dễ mở rộng
            \item Kết hợp deep learning với domain knowledge
        \end{itemize}
    \end{block}
    
    \vspace{0.5cm}
    
    \centering
    {\Large\textbf{Cảm ơn thầy/cô và các bạn đã lắng nghe!}}
    
    \vspace{0.3cm}
    
    {\small\faGithub\ \texttt{github.com/tam0918/banana-quality-grading}}
\end{frame}

% ==============================================================================
% Q&A
% ==============================================================================
\begin{frame}[plain]
    \centering
    \vspace{2cm}
    
    {\Huge\textbf{Q \& A}}
    
    \vspace{1cm}
    
    {\Large Câu hỏi \& Thảo luận}
    
    \vspace{1.5cm}
    
    \includegraphics[height=1.2cm]{images/hus_logo.png}
\end{frame}

% ==============================================================================
% BACKUP SLIDES
% ==============================================================================
\appendix

\begin{frame}{Backup: Chi tiết BananaAnalyzer}
    \vspace{0.3cm}
    \begin{table}[h]
        \small
        \centering
        \begin{tabular}{lcccc}
            \toprule
            \textbf{Feature} & \textbf{Unripe} & \textbf{Export} & \textbf{Overripe} & \textbf{Defective} \\
            \midrule
            yellow\_ratio & 0.1--0.3 & 0.6--0.8 & 0.4--0.6 & 0.2--0.4 \\
            green\_ratio & 0.5--0.7 & 0.05--0.15 & 0.05--0.1 & 0.0--0.1 \\
            brown\_ratio & 0.0--0.05 & 0.05--0.1 & 0.2--0.4 & 0.3--0.5 \\
            black\_ratio & 0.0 & 0.0--0.02 & 0.02--0.1 & 0.1--0.3 \\
            spot\_count & 0--2 & 2--5 & 10--30 & 30--100+ \\
            \bottomrule
        \end{tabular}
    \end{table}
    
    \vspace{0.5cm}
    {\small\textit{Các ngưỡng giá trị điển hình cho từng loại chất lượng}}
\end{frame}

\begin{frame}{Backup: So sánh với nghiên cứu khác}
    \vspace{0.3cm}
    \begin{table}[h]
        \centering
        \begin{tabular}{lccp{4.5cm}}
            \toprule
            \textbf{Nghiên cứu} & \textbf{Accuracy} & \textbf{Classes} & \textbf{Ghi chú} \\
            \midrule
            \textbf{Ours} & \textbf{98.75\%} & 4 & Real-time, 2-stage pipeline \\
            Mendoza (2004) & 93\% & 7 & Thống kê màu truyền thống \\
            Mazen (2019) & 96.7\% & 3 & ANN, xử lý offline \\
            Kaggle baseline & $\sim$95\% & 4 & MobileNetV2 \\
            \bottomrule
        \end{tabular}
    \end{table}
    
    \vspace{0.5cm}
    {\small\textit{So sánh với các công trình liên quan trong lĩnh vực}}
\end{frame}

\end{document}
