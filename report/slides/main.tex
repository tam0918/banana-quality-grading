% ==============================================================================
% SLIDE BÁO CÁO: ỨNG DỤNG THỊ GIÁC MÁY TÍNH TRONG PHÂN LOẠI CHẤT LƯỢNG CHUỐI
% ==============================================================================
\documentclass[aspectratio=169,10pt]{beamer}

% === PACKAGES ===
\usepackage[utf8]{inputenc}
\usepackage[vietnamese]{babel}
\usepackage{graphicx}
\usepackage{booktabs}
\usepackage{amsmath}
\usepackage{tikz}
\usepackage{pgfplots}
\usepackage{subcaption}
\usepackage{fontawesome5}

% === THEME ===
\usetheme{Madrid}
\usecolortheme{whale}
\setbeamertemplate{navigation symbols}{}
\setbeamertemplate{footline}[frame number]

% === COLORS ===
\definecolor{bananaYellow}{RGB}{255, 225, 53}
\definecolor{bananaGreen}{RGB}{138, 180, 71}
\definecolor{bananaBrown}{RGB}{139, 90, 43}
\definecolor{primaryBlue}{RGB}{0, 82, 147}

\setbeamercolor{title}{fg=white,bg=primaryBlue}
\setbeamercolor{frametitle}{fg=white,bg=primaryBlue}
\setbeamercolor{block title}{fg=white,bg=primaryBlue}
\setbeamercolor{structure}{fg=primaryBlue}

% === TITLE INFO ===
\title[Phân loại chất lượng Chuối]{Ứng dụng Thị giác Máy tính trong\\Phân loại Chất lượng Chuối}
\subtitle{Banana Quality Grading using Computer Vision}
\author{Sinh viên thực hiện}
\institute{Trường Đại học}
\date{Tháng 01/2026}

% === LOGO ===
% \logo{\includegraphics[height=0.8cm]{images/logo.png}}

\begin{document}

% ==============================================================================
% TITLE SLIDE
% ==============================================================================
\begin{frame}
    \titlepage
\end{frame}

% ==============================================================================
% OUTLINE
% ==============================================================================
\begin{frame}{Nội dung trình bày}
    \tableofcontents
\end{frame}

% ==============================================================================
% SECTION 1: GIỚI THIỆU
% ==============================================================================
\section{Giới thiệu}

\begin{frame}{Đặt vấn đề}
    \begin{columns}
        \column{0.55\textwidth}
        \textbf{Thách thức trong ngành chuối:}
        \begin{itemize}
            \item Phân loại thủ công tốn nhân lực
            \item Thiếu tính nhất quán trong đánh giá
            \item Khó kiểm soát chất lượng quy mô lớn
            \item Chuối là trái cây xuất khẩu quan trọng của Việt Nam
        \end{itemize}
        
        \vspace{0.5cm}
        \textbf{Giải pháp đề xuất:}
        \begin{itemize}
            \item Ứng dụng AI \& Computer Vision
            \item Phân loại tự động, real-time
            \item Hỗ trợ quyết định cho người dùng
        \end{itemize}
        
        \column{0.4\textwidth}
        \centering
        \includegraphics[width=\textwidth]{images/train_batch0.jpg}
        {\small\textit{Các loại chuối với độ chín khác nhau}}
    \end{columns}
\end{frame}

\begin{frame}{Mục tiêu nghiên cứu}
    \begin{block}{Mục tiêu chính}
        Xây dựng hệ thống phân loại chất lượng chuối \textbf{real-time} sử dụng YOLOv8
    \end{block}
    
    \vspace{0.3cm}
    
    \begin{columns}
        \column{0.25\textwidth}
        \centering
        \textcolor{bananaGreen}{\faLeaf}\\[0.2cm]
        \textbf{Unripe}\\
        Chuối xanh\\
        {\small Chưa thu hoạch}
        
        \column{0.25\textwidth}
        \centering
        \textcolor{bananaYellow}{\faStar}\\[0.2cm]
        \textbf{Export}\\
        Chín vừa\\
        {\small Xuất khẩu}
        
        \column{0.25\textwidth}
        \centering
        \textcolor{bananaBrown}{\faExclamationTriangle}\\[0.2cm]
        \textbf{Overripe}\\
        Quá chín\\
        {\small Bán gấp}
        
        \column{0.25\textwidth}
        \centering
        \textcolor{red}{\faTimesCircle}\\[0.2cm]
        \textbf{Defective}\\
        Hỏng/Thối\\
        {\small Loại bỏ}
    \end{columns}
\end{frame}

% ==============================================================================
% SECTION 2: PHƯƠNG PHÁP
% ==============================================================================
\section{Phương pháp}

\begin{frame}{Kiến trúc hệ thống 2-Stage Pipeline}
    \centering
    \begin{tikzpicture}[scale=0.85, transform shape,
        block/.style={rectangle, draw, fill=blue!20, text width=2.2cm, text centered, rounded corners, minimum height=1cm},
        arrow/.style={->, thick}
    ]
        % Input
        \node[block, fill=green!30] (input) at (0,0) {Video Frame\\(Webcam)};
        
        % Detector
        \node[block] (detector) at (3,0) {YOLO\\Detector\\(COCO)};
        
        % Classifier
        \node[block] (classifier) at (6,0) {YOLO\\Classifier\\(Fine-tuned)};
        
        % Analyzer
        \node[block, fill=orange!30] (analyzer) at (9,0) {Banana\\Analyzer\\(HSV/LAB)};
        
        % Output
        \node[block, fill=red!30] (output) at (12,0) {Kết quả\\Phân loại};
        
        % Arrows
        \draw[arrow] (input) -- (detector);
        \draw[arrow] (detector) -- node[above] {\small crop} (classifier);
        \draw[arrow] (classifier) -- (analyzer);
        \draw[arrow] (analyzer) -- (output);
    \end{tikzpicture}
    
    \vspace{0.5cm}
    
    \begin{columns}
        \column{0.5\textwidth}
        \textbf{Stage 1: Detection}
        \begin{itemize}
            \item COCO pretrained YOLOv8n
            \item Localization chuối trong frame
            \item Không cần annotation bbox
        \end{itemize}
        
        \column{0.5\textwidth}
        \textbf{Stage 2: Classification}
        \begin{itemize}
            \item YOLOv8n-cls fine-tuned
            \item Phân loại 4 class độ chín
            \item + BananaAnalyzer refinement
        \end{itemize}
    \end{columns}
\end{frame}

\begin{frame}{Tại sao chọn 2-Stage Pipeline?}
    \begin{alertblock}{Constraint → Opportunity}
        Dataset Kaggle chỉ có \textbf{classification labels}, không có bounding box!
    \end{alertblock}
    
    \vspace{0.3cm}
    
    \begin{columns}
        \column{0.5\textwidth}
        \textbf{Giải pháp thông minh:}
        \begin{enumerate}
            \item Dùng COCO detector có sẵn (class ``banana'')
            \item Fine-tune classifier riêng trên Kaggle data
            \item Kết hợp với BananaAnalyzer cho edge cases
        \end{enumerate}
        
        \column{0.5\textwidth}
        \textbf{Lợi ích:}
        \begin{itemize}
            \item[\faCheck] Không cần label bbox thủ công
            \item[\faCheck] Tận dụng pretrained knowledge
            \item[\faCheck] Có thể cải tiến từng component độc lập
            \item[\faCheck] Flexibility cao
        \end{itemize}
    \end{columns}
\end{frame}

\begin{frame}{Module BananaAnalyzer}
    \begin{columns}
        \column{0.45\textwidth}
        \textbf{Đặc trưng trích xuất:}
        \begin{table}[h]
            \small
            \begin{tabular}{ll}
                \toprule
                \textbf{Feature} & \textbf{Ý nghĩa} \\
                \midrule
                yellow\_ratio & Độ chín \\
                green\_ratio & Chưa chín \\
                brown\_ratio & Quá chín \\
                black\_ratio & Hỏng/thối \\
                spot\_count & Số đốm \\
                quality\_score & Điểm tổng hợp \\
                \bottomrule
            \end{tabular}
        \end{table}
        
        \column{0.5\textwidth}
        \textbf{Feature Refinement:}
        \begin{itemize}
            \item Khi classifier không chắc chắn (confidence < 50\%)
            \item Khi phát hiện dấu hiệu defective:
            \begin{itemize}
                \item black\_ratio > 0.15
                \item brown\_ratio > 0.3 + spot\_count > 10
            \end{itemize}
            \item Sửa $\sim$10-15\% edge cases
        \end{itemize}
    \end{columns}
\end{frame}

% ==============================================================================
% SECTION 3: DỮ LIỆU VÀ HUẤN LUYỆN
% ==============================================================================
\section{Dữ liệu \& Huấn luyện}

\begin{frame}{Dataset: Kaggle Banana Classification}
    \begin{columns}
        \column{0.45\textwidth}
        \begin{table}[h]
            \small
            \begin{tabular}{lr}
                \toprule
                \textbf{Thuộc tính} & \textbf{Giá trị} \\
                \midrule
                Tổng số ảnh & $\sim$13,500 \\
                Số class & 4 \\
                Dung lượng & 227 MB \\
                License & MIT \\
                \midrule
                Training & 80\% \\
                Validation & 10\% \\
                Test & 10\% \\
                \bottomrule
            \end{tabular}
        \end{table}
        
        \textbf{4 Classes:}
        \begin{enumerate}
            \item Unripe (Xanh)
            \item Ripe (Chín)
            \item Overripe (Quá chín)
            \item Rotten (Thối)
        \end{enumerate}
        
        \column{0.5\textwidth}
        \centering
        \includegraphics[width=0.9\textwidth]{images/train_batch1.jpg}
        {\small\textit{Ví dụ ảnh từ training set}}
    \end{columns}
\end{frame}

\begin{frame}{Quá trình huấn luyện}
    \begin{columns}
        \column{0.45\textwidth}
        \textbf{Cấu hình:}
        \begin{table}[h]
            \small
            \begin{tabular}{ll}
                \toprule
                Base model & yolov8n-cls.pt \\
                Image size & 416×416 \\
                Epochs & 50 \\
                Batch size & Auto \\
                Optimizer & AdamW \\
                Device & GPU (CUDA) \\
                \bottomrule
            \end{tabular}
        \end{table}
        
        \textbf{Thời gian:} $\sim$71 phút
        
        \column{0.5\textwidth}
        \centering
        \includegraphics[width=\textwidth]{images/results.png}
        {\small\textit{Training metrics qua 50 epochs}}
    \end{columns}
\end{frame}

% ==============================================================================
% SECTION 4: KẾT QUẢ
% ==============================================================================
\section{Kết quả}

\begin{frame}{Giai đoạn 1: Kết quả Classifier ``Thô''}
    \begin{columns}
        \column{0.4\textwidth}
        \begin{block}{Metrics trên Validation Set}
            \begin{itemize}
                \item \textbf{Top-1 Accuracy: 98.75\%}
                \item Top-5 Accuracy: 100\%
                \item Training Loss: 0.0096
                \item Val Loss: 0.0752
            \end{itemize}
        \end{block}
        
        \vspace{0.3cm}
        \textbf{Nhận định:}
        \begin{itemize}
            \item Model hội tụ nhanh ($\sim$15 epochs)
            \item Transfer learning hiệu quả
            \item \textcolor{red}{Nhưng: chỉ trên clean data!}
        \end{itemize}
        
        \column{0.55\textwidth}
        \centering
        \includegraphics[width=\textwidth]{images/confusion_matrix_normalized.png}
        {\small\textit{Confusion Matrix (normalized)}}
    \end{columns}
\end{frame}

\begin{frame}{Phân tích Confusion Matrix}
    \begin{columns}
        \column{0.55\textwidth}
        \includegraphics[width=\textwidth]{images/confusion_matrix.png}
        
        \column{0.4\textwidth}
        \textbf{Nhận định quan trọng:}
        \begin{itemize}
            \item Model học được \textbf{thứ tự chín}
            \item Lỗi chỉ xảy ra giữa class liền kề:
            \begin{itemize}
                \item ripe $\leftrightarrow$ overripe
                \item overripe $\leftrightarrow$ rotten
            \end{itemize}
            \item \textbf{Không có} nhầm lẫn cực đoan (unripe $\leftrightarrow$ rotten)
        \end{itemize}
        
        \vspace{0.3cm}
        $\Rightarrow$ Feature space phù hợp với bản chất tự nhiên của quá trình chín!
    \end{columns}
\end{frame}

\begin{frame}{Giai đoạn 2: Pipeline Hoàn chỉnh}
    \begin{columns}
        \column{0.45\textwidth}
        \textbf{Test trên 15 ảnh thực tế:}
        \begin{table}[h]
            \small
            \begin{tabular}{lcc}
                \toprule
                \textbf{Category} & \textbf{Frame} & \textbf{Instance} \\
                \midrule
                Export & 4 & 7 \\
                Unripe & --- & 9 \\
                Overripe & 3 & 3 \\
                Defective & 4 & 4 \\
                None & 4 & --- \\
                \bottomrule
            \end{tabular}
        \end{table}
        
        \textbf{Kết quả:}
        \begin{itemize}
            \item[\faCheck] Negative samples xử lý đúng
            \item[\faCheck] Multi-object detection OK
            \item[\faCheck] Feature refinement hoạt động
        \end{itemize}
        
        \column{0.5\textwidth}
        \centering
        \includegraphics[width=0.95\textwidth]{images/real_e2e.jpg}
        {\small\textit{Kết quả test end-to-end}}
    \end{columns}
\end{frame}

\begin{frame}{Ví dụ kết quả Pipeline}
    \begin{columns}
        \column{0.33\textwidth}
        \centering
        \includegraphics[width=\textwidth]{images/pipeline_results/3.jpg}
        {\small Overripe (100\%)}
        
        \column{0.33\textwidth}
        \centering
        \includegraphics[width=\textwidth]{images/pipeline_results/5.jpg}
        {\small Export (100\%)}
        
        \column{0.33\textwidth}
        \centering
        \includegraphics[width=\textwidth]{images/pipeline_results/7.jpg}
        {\small Defective (99.96\%)}
    \end{columns}
    
    \vspace{0.5cm}
    
    \begin{columns}
        \column{0.48\textwidth}
        \centering
        \includegraphics[width=0.85\textwidth]{images/pipeline_results/14.jpg}
        {\small Nải 6 quả: 4 Unripe + 2 Export}
        
        \column{0.48\textwidth}
        \centering
        \includegraphics[width=0.85\textwidth]{images/pipeline_results/15.jpg}
        {\small Nải 6 quả: Overall = Defective}
    \end{columns}
\end{frame}

\begin{frame}{Hiệu năng Real-time}
    \begin{columns}
        \column{0.5\textwidth}
        \begin{table}[h]
            \small
            \begin{tabular}{lcc}
                \toprule
                \textbf{Hardware} & \textbf{Total} & \textbf{FPS} \\
                \midrule
                RTX 3060 & $\sim$13 ms & 40-75 \\
                GTX 1650 & $\sim$23 ms & 30-40 \\
                Intel i7 (CPU) & $\sim$70 ms & 12-15 \\
                Intel i5 (CPU) & $\sim$120 ms & 8-10 \\
                \bottomrule
            \end{tabular}
        \end{table}
        
        \textbf{Kết luận:}
        \begin{itemize}
            \item GPU mid-range: \textbf{Real-time mượt mà}
            \item CPU: Chấp nhận được cho demo
        \end{itemize}
        
        \column{0.45\textwidth}
        \begin{block}{Công thức Latency}
            \small
            $\text{Latency} = T_{\text{capture}} + T_{\text{det}} + T_{\text{cls}} + T_{\text{analyzer}} + T_{\text{render}}$
        \end{block}
        
        \vspace{0.3cm}
        Trên GPU mid-range:\\
        $\text{Latency}_{\text{total}} \approx 50-80$ ms
    \end{columns}
\end{frame}

\begin{frame}{So sánh: Model ``Thô'' vs Pipeline Hoàn chỉnh}
    \begin{table}[h]
        \begin{tabular}{lcc}
            \toprule
            \textbf{Tiêu chí} & \textbf{Giai đoạn 1} & \textbf{Giai đoạn 2} \\
            \midrule
            Dữ liệu đánh giá & Kaggle val set & Ảnh thực tế \\
            Có localization & \textcolor{red}{\faTimes} & \textcolor{green}{\faCheck} \\
            Xử lý nhiều quả & \textcolor{red}{\faTimes} & \textcolor{green}{\faCheck} \\
            Feature refinement & \textcolor{red}{\faTimes} & \textcolor{green}{\faCheck} \\
            Edge case handling & \textcolor{red}{\faTimes} & \textcolor{green}{\faCheck} \\
            Accuracy báo cáo & 98.75\% (clean) & Phụ thuộc \\
            \bottomrule
        \end{tabular}
    \end{table}
    
    \begin{alertblock}{Bài học quan trọng}
        \centering
        \textbf{Accuracy trên validation set $\neq$ Hiệu năng thực tế!}
    \end{alertblock}
\end{frame}

% ==============================================================================
% SECTION 5: DEMO & KẾT LUẬN
% ==============================================================================
\section{Demo \& Kết luận}

\begin{frame}{Demo hệ thống}
    \centering
    \begin{tikzpicture}
        \node[draw, rounded corners, fill=blue!10, minimum width=10cm, minimum height=5cm] (demo) {};
        \node at (demo.center) {\Large\textbf{DEMO VIDEO / LIVE DEMO}};
        \node[below] at (demo.south) {\small Chạy \texttt{python main.py} để khởi động ứng dụng};
    \end{tikzpicture}
    
    \vspace{0.5cm}
    
    \begin{columns}
        \column{0.5\textwidth}
        \textbf{Tính năng chính:}
        \begin{itemize}
            \item Real-time webcam processing
            \item Bounding box + label tiếng Việt
            \item Hiển thị confidence + quality score
        \end{itemize}
        
        \column{0.5\textwidth}
        \textbf{Yêu cầu:}
        \begin{itemize}
            \item Python 3.9+
            \item Webcam
            \item GPU (khuyến nghị)
        \end{itemize}
    \end{columns}
\end{frame}

\begin{frame}{Đóng góp chính}
    \begin{enumerate}
        \item \textbf{Kiến trúc 2-stage hiệu quả:}
        \begin{itemize}
            \item Giải quyết thiếu bbox annotation
            \item Tận dụng COCO pretrained + Kaggle classification data
        \end{itemize}
        
        \item \textbf{Module BananaAnalyzer:}
        \begin{itemize}
            \item Kết hợp deep learning + domain knowledge (HSV/LAB)
            \item Feature refinement cho edge cases
        \end{itemize}
        
        \item \textbf{Hệ thống real-time hoàn chỉnh:}
        \begin{itemize}
            \item 98.75\% accuracy trên validation
            \item 40-75 FPS trên GPU mid-range
            \item Giao diện tiếng Việt thân thiện
        \end{itemize}
    \end{enumerate}
\end{frame}

\begin{frame}{Hạn chế \& Hướng phát triển}
    \begin{columns}
        \column{0.45\textwidth}
        \textbf{Hạn chế:}
        \begin{itemize}
            \item Phụ thuộc COCO detector
            \item Dataset nền đơn giản
            \item Ranh giới mờ overripe/rotten
            \item Chưa test nhiều giống chuối
        \end{itemize}
        
        \column{0.5\textwidth}
        \textbf{Hướng phát triển:}
        \begin{itemize}
            \item Train detector riêng
            \item Export TFLite cho Android
            \item Mở rộng sang trái cây khác
            \item Dự đoán shelf-life
            \item Explainable AI (Grad-CAM)
        \end{itemize}
    \end{columns}
\end{frame}

\begin{frame}{Kết luận}
    \begin{block}{Tổng kết}
        Đề tài đã xây dựng thành công hệ thống phân loại chất lượng chuối real-time với:
        \begin{itemize}
            \item \textbf{98.75\% Top-1 Accuracy} trên validation set
            \item \textbf{40-75 FPS} trên GPU mid-range
            \item Pipeline 2-stage linh hoạt, dễ mở rộng
        \end{itemize}
    \end{block}
    
    \vspace{0.3cm}
    
    \begin{center}
        \Large\textbf{Cảm ơn thầy/cô và các bạn đã lắng nghe!}
        
        \vspace{0.5cm}
        
        \faGithub\ \texttt{github.com/[username]/banana-quality-grading}
    \end{center}
\end{frame}

% ==============================================================================
% Q&A
% ==============================================================================
\begin{frame}{}
    \centering
    \Huge\textbf{Q \& A}
    
    \vspace{1cm}
    
    \Large Câu hỏi \& Thảo luận
\end{frame}

% ==============================================================================
% BACKUP SLIDES
% ==============================================================================
\appendix

\begin{frame}{Backup: Chi tiết BananaAnalyzer}
    \begin{table}[h]
        \small
        \begin{tabular}{lcccc}
            \toprule
            \textbf{Feature} & \textbf{Unripe} & \textbf{Export} & \textbf{Overripe} & \textbf{Defective} \\
            \midrule
            yellow\_ratio & 0.1-0.3 & 0.6-0.8 & 0.4-0.6 & 0.2-0.4 \\
            green\_ratio & 0.5-0.7 & 0.05-0.15 & 0.05-0.1 & 0.0-0.1 \\
            brown\_ratio & 0.0-0.05 & 0.05-0.1 & 0.2-0.4 & 0.3-0.5 \\
            black\_ratio & 0.0 & 0.0-0.02 & 0.02-0.1 & 0.1-0.3 \\
            spot\_count & 0-2 & 2-5 & 10-30 & 30-100+ \\
            \bottomrule
        \end{tabular}
    \end{table}
\end{frame}

\begin{frame}{Backup: So sánh với nghiên cứu khác}
    \begin{table}[h]
        \begin{tabular}{lccp{4cm}}
            \toprule
            \textbf{Nghiên cứu} & \textbf{Accuracy} & \textbf{Classes} & \textbf{Ghi chú} \\
            \midrule
            \textbf{Ours} & \textbf{98.75\%} & 4 & Real-time, 2-stage \\
            Mendoza (2004) & 93\% & 7 & Thống kê màu \\
            Mazen (2019) & 96.7\% & 3 & ANN, offline \\
            Kaggle baseline & $\sim$95\% & 4 & MobileNetV2 \\
            \bottomrule
        \end{tabular}
    \end{table}
\end{frame}

\end{document}
