%!TEX encoding = UTF-8 Unicode
%!TEX program = xelatex
% ==============================================================================
% BÁO CÁO MÔN HỌC: THỊ GIÁC MÁY TÍNH
% Đề tài: Ứng dụng thị giác máy tính trong phân loại chất lượng Chuối
% Trường Đại học Khoa học Tự nhiên - ĐHQGHN
% ==============================================================================

\documentclass[12pt,a4paper]{report}

% ========================= PACKAGES =========================
\usepackage[utf8]{inputenc}
\usepackage[vietnamese]{babel}
\usepackage{fontspec}
\usepackage{amsmath,amssymb,amsfonts}
\usepackage{graphicx}
\usepackage{xcolor}
\usepackage{geometry}
\usepackage{fancyhdr}
\usepackage{titlesec}
\usepackage{tocloft}
\usepackage{hyperref}
\usepackage{listings}
\usepackage{algorithm}
\usepackage{algpseudocode}
\usepackage{booktabs}
\usepackage{multirow}
\usepackage{longtable}
\usepackage{caption}
\usepackage{subcaption}
\usepackage{float}
\usepackage{enumitem}
\usepackage{tikz}
\usepackage{pgfplots}
\usepackage{glossaries}
\usepackage{appendix}
\usepackage{pdfpages}
\usepackage{setspace}
\usepackage{graphicx}
\usepackage{geometry}
\usepackage{mathptmx} % Font giống Times New Roman
\usepackage{tikz} % Để vẽ khung viền
\usetikzlibrary{calc}

% ========================= CẤU HÌNH TRANG =========================
\geometry{
    left=3cm,
    right=2cm,
    top=2.5cm,
    bottom=2.5cm
}

% Font configuration
\setmainfont{Times New Roman}

% Spacing
\onehalfspacing

% ========================= HYPERREF =========================
\hypersetup{
    colorlinks=true,
    linkcolor=blue!70!black,
    citecolor=green!50!black,
    urlcolor=blue!80!black,
    pdftitle={Ứng dụng thị giác máy tính trong phân loại chất lượng Chuối},
    pdfauthor={Lường Văn Tâm, Khương Thanh Tín}
}

% ========================= LISTINGS CONFIG =========================
\definecolor{codegreen}{rgb}{0,0.6,0}
\definecolor{codegray}{rgb}{0.5,0.5,0.5}
\definecolor{codepurple}{rgb}{0.58,0,0.82}
\definecolor{backcolour}{rgb}{0.95,0.95,0.92}

\lstdefinestyle{pythonstyle}{
    backgroundcolor=\color{backcolour},
    commentstyle=\color{codegreen},
    keywordstyle=\color{blue},
    numberstyle=\tiny\color{codegray},
    stringstyle=\color{codepurple},
    basicstyle=\ttfamily\footnotesize,
    breakatwhitespace=false,
    breaklines=true,
    captionpos=b,
    keepspaces=true,
    numbers=left,
    numbersep=5pt,
    showspaces=false,
    showstringspaces=false,
    showtabs=false,
    tabsize=2,
    language=Python
}

\lstset{style=pythonstyle}

% ========================= HEADERS & FOOTERS =========================
\pagestyle{fancy}
\fancyhf{}
\fancyhead[L]{\leftmark}
\fancyhead[R]{\thepage}
\renewcommand{\headrulewidth}{0.5pt}

% ========================= CHAPTER/SECTION FORMATTING =========================
\titleformat{\chapter}[display]
    {\normalfont\huge\bfseries}{\chaptertitlename\ \thechapter}{20pt}{\Huge}
\titlespacing*{\chapter}{0pt}{-20pt}{40pt}

% ========================= TABLE OF CONTENTS =========================
\renewcommand{\contentsname}{MỤC LỤC}
\renewcommand{\listfigurename}{DANH MỤC HÌNH ẢNH}
\renewcommand{\listtablename}{DANH MỤC BẢNG BIỂU}

% ========================= CAPTION =========================
\captionsetup{font=small, labelfont=bf}

% ========================= GLOSSARY =========================
\makeglossaries

% Định nghĩa các từ viết tắt
\newacronym{cnn}{CNN}{Convolutional Neural Network - Mạng nơ-ron tích chập}
\newacronym{yolo}{YOLO}{You Only Look Once - Thuật toán nhận diện đối tượng thời gian thực}
\newacronym{hsv}{HSV}{Hue-Saturation-Value - Không gian màu}
\newacronym{lab}{LAB}{Lightness-A-B Color Space - Không gian màu CIE L*a*b*}
\newacronym{roi}{ROI}{Region of Interest - Vùng quan tâm}
\newacronym{bbox}{BBox}{Bounding Box - Hộp giới hạn}
\newacronym{fps}{FPS}{Frames Per Second - Số khung hình mỗi giây}
\newacronym{gpu}{GPU}{Graphics Processing Unit - Bộ xử lý đồ họa}
\newacronym{cpu}{CPU}{Central Processing Unit - Bộ xử lý trung tâm}
\newacronym{api}{API}{Application Programming Interface - Giao diện lập trình ứng dụng}
\newacronym{clahe}{CLAHE}{Contrast Limited Adaptive Histogram Equalization}
\newacronym{nms}{NMS}{Non-Maximum Suppression - Triệt tiêu phi cực đại}
\newacronym{map}{mAP}{mean Average Precision - Độ chính xác trung bình}
\newacronym{iou}{IoU}{Intersection over Union - Tỉ lệ giao trên hợp}
\newacronym{ui}{UI}{User Interface - Giao diện người dùng}

% ========================= DOCUMENT =========================
\begin{document}

% ========================= TRANG BÌA =========================

\begin{titlepage}
    % --- VẼ KHUNG VIỀN ---
    \begin{tikzpicture}[remember picture, overlay]
        \draw [line width=3pt]
            ($ (current page.north west) + (2.0cm,-2.0cm) $)
            rectangle
            ($ (current page.south east) + (-1.5cm,2.0cm) $);
        \draw [line width=1pt]
            ($ (current page.north west) + (2.15cm,-2.15cm) $)
            rectangle
            ($ (current page.south east) + (-1.65cm,2.15cm) $);
    \end{tikzpicture}

    \begin{center}
        % --- PHẦN HEADER ---
        \vspace{-0.5cm}
        \textbf{\large ĐẠI HỌC QUỐC GIA HÀ NỘI}\\
        \textbf{\large TRƯỜNG ĐẠI HỌC KHOA HỌC TỰ NHIÊN}\\
        \textbf{\large KHOA VẬT LÝ}\\
        \vspace{0.3cm}
        \rule{6cm}{1pt} % Dòng kẻ ngắn dưới tên khoa
        
        \vspace{1.5cm}
        
        % --- LOGO ---
        % Chú ý: Đảm bảo bạn có file logo đúng đường dẫn
        \includegraphics[width=4cm]{images/hus_logo.png} 
        
        \vspace{1.5cm}
        
        % --- TÊN BÁO CÁO ---
        \textbf{\LARGE BÁO CÁO CUỐI KỲ}\\
        \vspace{0.3cm}
        \textbf{\Large Môn học: Thị giác máy tính}\\
        \vspace{1.5cm}
        
        % --- TÊN ĐỀ TÀI (Nổi bật nhất) ---
        \textbf{\Huge ỨNG DỤNG THỊ GIÁC MÁY TÍNH}\\
        \vspace{0.3cm}
        \textbf{\Huge TRONG PHÂN LOẠI CHẤT LƯỢNG CHUỐI}
        
        \vspace{2.5cm}
        
        % --- THÔNG TIN SINH VIÊN/GV (Căn chỉnh lại cho gọn) ---
        \begin{table}[h]
            \centering
            \begin{tabular}{r l}
                \textbf{Giảng viên hướng dẫn:} & PGS. TS. Phạm Tiến Lâm \\
                                               & Vi Anh Quân \\[0.5cm]
                \textbf{Sinh viên thực hiện:}  & 1. Lường Văn Tâm (22001349) \\
                                               & 2. Khương Thanh Tín (22001358) \\[0.5cm]
                \textbf{Lớp:}                  & K67 Kỹ thuật điện tử và Tin học
            \end{tabular}
        \end{table}
        
        \vfill % Tự động đẩy phần dưới cùng xuống đáy
        
        % --- FOOTER ---
        \textbf{Hà Nội -- 2026}
    \end{center}
\end{titlepage}

\clearpage

% ========================= LỜI CẢM ƠN =========================
% ==============================================================================
% LỜI CẢM ƠN
% ==============================================================================

\chapter*{LỜI CẢM ƠN}
\addcontentsline{toc}{chapter}{LỜI CẢM ƠN}

\vspace{0.5cm}

Trong quá trình học tập và nghiên cứu, chúng em đã nhận được sự hướng dẫn tận tình và những góp ý quý báu từ nhiều thầy cô và bạn bè. Nhân dịp hoàn thành báo cáo này, chúng em xin được bày tỏ lòng biết ơn sâu sắc.

\vspace{0.5cm}

Trước hết, chúng em xin gửi lời cảm ơn chân thành nhất đến \textbf{PGS. TS. Phạm Tiến Lâm} và \textbf{Vi Anh Quân} --- những người thầy đã trực tiếp hướng dẫn chúng em trong suốt quá trình thực hiện đề tài. Sự chỉ dẫn tận tâm, những nhận xét sắc bén và tinh thần khoa học nghiêm túc của các thầy đã giúp chúng em định hướng rõ ràng hơn trong nghiên cứu, đồng thời khơi gợi những suy ngẫm sâu sắc về bản chất của các thuật toán học máy và thị giác máy tính.

\vspace{0.5cm}

Chúng em cũng xin cảm ơn các thầy cô trong \textbf{Khoa Vật Lý}, \textbf{Trường Đại học Khoa học Tự nhiên}, \textbf{Đại học Quốc gia Hà Nội} đã trang bị cho chúng em nền tảng kiến thức vững chắc về vật lý, xử lý tín hiệu và trí tuệ nhân tạo. Đặc biệt, môn học \textit{Thị giác máy tính} đã mở ra cho chúng em một chân trời mới về khả năng của máy móc trong việc ``nhìn'' và ``hiểu'' thế giới xung quanh.

\vspace{0.5cm}

Cuối cùng, chúng em xin gửi lời cảm ơn đến gia đình và bạn bè --- những người đã luôn động viên, ủng hộ chúng em trong suốt quá trình học tập và nghiên cứu. Sự khích lệ của mọi người là động lực to lớn giúp chúng em vượt qua những khó khăn và hoàn thành tốt công việc được giao.

\vspace{0.5cm}

Mặc dù đã cố gắng hết sức, báo cáo này chắc chắn không tránh khỏi những thiếu sót. Chúng em rất mong nhận được những ý kiến đóng góp từ các thầy cô và bạn đọc để hoàn thiện hơn trong tương lai.

\vspace{1cm}

\begin{flushright}
\textit{Hà Nội, tháng 01 năm 2026}\\[0.5cm]
\textbf{Nhóm sinh viên thực hiện}\\[0.3cm]
Lường Văn Tâm\\
Khương Thanh Tín
\end{flushright}

\clearpage


% ========================= MỤC LỤC =========================
\tableofcontents
\clearpage

% ========================= DANH MỤC HÌNH ẢNH =========================
\listoffigures
\addcontentsline{toc}{chapter}{DANH MỤC HÌNH ẢNH}
\clearpage

% ========================= DANH MỤC BẢNG BIỂU =========================
\listoftables
\addcontentsline{toc}{chapter}{DANH MỤC BẢNG BIỂU}
\clearpage

% ========================= DANH MỤC TỪ VIẾT TẮT =========================
\printglossary[type=\acronymtype, title=DANH MỤC TỪ VIẾT TẮT]
\addcontentsline{toc}{chapter}{DANH MỤC TỪ VIẾT TẮT}
\clearpage

% ========================= NỘI DUNG CHÍNH =========================
% ==============================================================================
% CHƯƠNG 1: GIỚI THIỆU
% ==============================================================================

\chapter{GIỚI THIỆU}
\label{chap:introduction}

\section{Đặt vấn đề}

Trong bối cảnh nền nông nghiệp hiện đại đang chuyển mình mạnh mẽ theo hướng công nghệ hóa và tự động hóa, việc ứng dụng trí tuệ nhân tạo (\textit{Artificial Intelligence}) và thị giác máy tính (\textit{Computer Vision}) vào các quy trình sản xuất và kiểm định chất lượng nông sản đã trở thành một xu thế tất yếu. Không chỉ đơn thuần là câu chuyện về hiệu suất kinh tế, đây còn là vấn đề mang tính triết học sâu xa: liệu máy móc có thể ``nhìn'' và ``hiểu'' thế giới tự nhiên theo cách mà con người vẫn làm?

Chuối (\textit{Musa spp.}) là một trong những loại trái cây được tiêu thụ rộng rãi nhất trên thế giới, đóng vai trò quan trọng trong chuỗi cung ứng thực phẩm toàn cầu. Tại Việt Nam, với sản lượng hàng triệu tấn mỗi năm, ngành công nghiệp chuối đang đối mặt với thách thức lớn trong việc phân loại và kiểm soát chất lượng. Phương pháp truyền thống dựa vào kinh nghiệm và thị giác của con người không chỉ tốn kém về nhân lực mà còn thiếu tính nhất quán, dẫn đến những sai sót đáng kể trong khâu phân loại.

Từ góc nhìn của một nhà nghiên cứu học máy lý thuyết, bài toán phân loại chất lượng chuối không đơn thuần là một ứng dụng công nghệ --- mà còn là cơ hội để khám phá ranh giới giữa nhận thức của con người và khả năng học tập của máy. Câu hỏi then chốt không chỉ là ``làm sao để máy phân loại chính xác'' mà còn là ``tại sao mô hình học được những đặc trưng này''.

\textbf{Bối cảnh dữ liệu:} Một trong những thách thức thực tế của bài toán là việc thiếu các dataset công khai chất lượng cao về độ chín chuối. May mắn thay, tập dữ liệu \textbf{Banana Classification Dataset} trên Kaggle \cite{kaggle_banana_classification} đã cung cấp một nguồn dữ liệu đáng tin cậy với $\sim$13,500 ảnh được phân loại vào 4 class (unripe, ripe, overripe, rotten), cùng với giấy phép MIT cho phép sử dụng học thuật và thương mại.

\section{Mục tiêu nghiên cứu}

Đề tài này hướng đến các mục tiêu cụ thể sau:

\begin{enumerate}[label=\textbf{\arabic*.}]
    \item \textbf{Xây dựng hệ thống phân loại chất lượng chuối theo thời gian thực}: Sử dụng kiến trúc 2 giai đoạn (two-stage pipeline) kết hợp \gls{yolo} detector và classifier để phát hiện và phân loại độ chín của chuối từ luồng video webcam.
    
    \item \textbf{Phát triển module phân tích hình ảnh nâng cao}: Dựa trên các nghiên cứu khoa học về xử lý ảnh, triển khai các kỹ thuật trích xuất đặc trưng màu sắc (HSV/LAB), hình thái học và kết cấu bề mặt để tăng cường độ tin cậy của kết quả phân loại.
    
    \item \textbf{Phân loại theo 4 cấp độ hành động}: Chia chuối thành 4 nhóm phục vụ mục đích thương mại:
    \begin{itemize}
        \item \textit{Unripe} (Xanh): Chưa thu hoạch
        \item \textit{Export} (Chín vừa): Phù hợp xuất khẩu
        \item \textit{Overripe} (Quá chín): Cần bán gấp hoặc tiêu thụ ngay
        \item \textit{Defective} (Hỏng/Bệnh): Cần loại bỏ
    \end{itemize}
    
    \item \textbf{Thiết kế giao diện người dùng thân thiện}: Xây dựng \gls{ui} bằng CustomTkinter với hỗ trợ tiếng Việt đầy đủ, hiển thị trực quan các kết quả phân loại và điểm chất lượng.
    
    \item \textbf{Đánh giá hiệu năng và độ chính xác}: Thực hiện các thí nghiệm để đo lường độ chính xác phân loại, tốc độ xử lý và khả năng ứng dụng thực tế của hệ thống.
\end{enumerate}

\section{Phạm vi nghiên cứu}

\subsection{Phạm vi về đối tượng}
\begin{itemize}
    \item Đối tượng chính: Quả chuối tiêu (Cavendish) --- giống chuối phổ biến nhất trong thương mại quốc tế.
    \item Trạng thái: Chuối đơn lẻ hoặc trong nải (nhiều quả trong cùng khung hình).
    \item Điều kiện chụp: Cầm chuối trên tay, ánh sáng vừa đủ (trong nhà hoặc ngoài trời râm).
\end{itemize}

\subsection{Phạm vi về kỹ thuật}
\begin{itemize}
    \item Sử dụng mô hình YOLOv8 (Ultralytics) cho cả detection và classification.
    \item Huấn luyện trên tập dữ liệu công khai từ Kaggle (Banana Ripeness Classification Dataset).
    \item Chạy inference trên CPU hoặc GPU (CUDA).
    \item Không sử dụng các dịch vụ API trả phí --- chỉ dùng pretrained weights open-source để fine-tune.
\end{itemize}

\section{Ý nghĩa khoa học và thực tiễn}

\subsection{Ý nghĩa khoa học}
Nghiên cứu này đóng góp vào lĩnh vực thị giác máy tính ứng dụng trong nông nghiệp thông qua:

\begin{itemize}
    \item Khảo sát và so sánh hiệu quả của kiến trúc 2 giai đoạn (detector + classifier) so với phương pháp end-to-end trong bài toán phân loại nông sản.
    
    \item Đề xuất phương pháp kết hợp deep learning với các kỹ thuật xử lý ảnh cổ điển (phân tích màu HSV, phát hiện đốm, đo texture) để tăng cường độ tin cậy cho trường hợp edge-case.
    
    \item Phân tích mối quan hệ giữa các đặc trưng thị giác (màu sắc, hình thái, kết cấu) với độ chín sinh học của chuối --- một bước tiến gần hơn đến việc hiểu ``tại sao'' mô hình đưa ra quyết định.
\end{itemize}

\subsection{Ý nghĩa thực tiễn}
Về mặt ứng dụng, hệ thống có thể được triển khai trong:

\begin{itemize}
    \item \textbf{Nhà kho/trạm phân loại}: Hỗ trợ công nhân kiểm tra nhanh chất lượng, giảm tải công việc thủ công.
    
    \item \textbf{Xuất khẩu nông sản}: Đảm bảo chỉ chuối đạt tiêu chuẩn ``export'' được đóng gói.
    
    \item \textbf{Siêu thị/cửa hàng}: Kiểm tra độ tươi của hàng hóa trưng bày.
    
    \item \textbf{Nông dân}: Ứng dụng di động (sau khi export TFLite) để đánh giá thu hoạch tại vườn.
\end{itemize}

\section{Cấu trúc báo cáo}

Báo cáo được tổ chức thành 6 chương với nội dung như sau:

\begin{itemize}
    \item \textbf{Chương 1 --- Giới thiệu}: Trình bày bối cảnh, mục tiêu, phạm vi và ý nghĩa của đề tài.
    
    \item \textbf{Chương 2 --- Cơ sở lý thuyết}: Khái quát các kiến thức nền tảng về thị giác máy tính, mạng nơ-ron tích chập, kiến trúc YOLO và các không gian màu.
    
    \item \textbf{Chương 3 --- Phương pháp thực hiện}: Mô tả chi tiết kiến trúc hệ thống 2-stage, module BananaAnalyzer và các kỹ thuật refinement.
    
    \item \textbf{Chương 4 --- Cài đặt và triển khai}: Trình bày quá trình thu thập dữ liệu, huấn luyện mô hình và phát triển giao diện.
    
    \item \textbf{Chương 5 --- Kết quả và đánh giá}: Phân tích kết quả theo 2 giai đoạn: (1) model ``thô'' trên validation set, (2) pipeline hoàn chỉnh trên dữ liệu thực tế.
    
    \item \textbf{Chương 6 --- Kết luận}: Tổng kết công việc, bài học kinh nghiệm và đề xuất hướng phát triển.
\end{itemize}

\textbf{Dòng chảy tư duy:} Mỗi chương được thiết kế để trả lời một câu hỏi cụ thể: \textit{Tại sao làm?} (Ch.1) $\rightarrow$ \textit{Dựa trên cơ sở gì?} (Ch.2) $\rightarrow$ \textit{Thiết kế như thế nào?} (Ch.3) $\rightarrow$ \textit{Triển khai ra sao?} (Ch.4) $\rightarrow$ \textit{Kết quả thế nào?} (Ch.5) $\rightarrow$ \textit{Rút ra điều gì?} (Ch.6).

\clearpage

% ==============================================================================
% CHƯƠNG 2: CƠ SỞ LÝ THUYẾT
% ==============================================================================

\chapter{CƠ SỞ LÝ THUYẾT}
\label{chap:theory}

Chương này trình bày nền tảng lý thuyết cho hệ thống phân loại chất lượng chuối. Thay vì liệt kê thuần túy các công thức, chúng tôi cố gắng xây dựng một bức tranh kết nối giữa các khái niệm, trả lời câu hỏi \textit{``tại sao''} đằng sau mỗi lựa chọn kỹ thuật.

\section{Tổng quan về Thị giác máy tính}

\subsection{Định nghĩa và vai trò}

Thị giác máy tính (\textit{Computer Vision}) là một lĩnh vực liên ngành của khoa học máy tính, nghiên cứu cách thức máy móc có thể thu nhận, xử lý và ``hiểu'' thông tin thị giác từ thế giới thực. Nếu nhìn từ góc độ triết học nhận thức, thị giác máy tính đặt ra câu hỏi cốt lõi: \textit{Liệu máy móc có thể nhìn thế giới như cách con người nhìn?}

Câu trả lời, theo góc nhìn của chúng tôi, không phải là ``có'' hay ``không'' tuyệt đối, mà nằm ở việc hiểu rằng máy và người ``nhìn'' theo những cách khác nhau nhưng có thể đạt được kết quả tương đương trong các tác vụ cụ thể. Đây chính là điểm giao thoa giữa lý thuyết học máy và ứng dụng thực tiễn.

\textbf{Một suy ngẫm về bản chất:} Khi một mạng nơ-ron ``nhận ra'' quả chuối chín, nó không thực sự ``hiểu'' khái niệm ``chín'' theo cách con người hiểu. Thay vào đó, nó học được một \textit{mapping} từ không gian pixel sang không gian nhãn, mà mapping này \textit{tình cờ} phù hợp với cách con người phân loại. Hiểu điều này giúp chúng ta đặt kỳ vọng đúng đắn về khả năng và giới hạn của hệ thống.

\subsection{Các bài toán cơ bản trong thị giác máy tính}

\begin{table}[H]
\centering
\caption{Các bài toán chính trong thị giác máy tính}
\label{tab:cv_problems}
\begin{tabular}{@{}llp{7cm}@{}}
\toprule
\textbf{STT} & \textbf{Bài toán} & \textbf{Mô tả} \\
\midrule
1 & Image Classification & Phân loại toàn bộ ảnh vào một trong các lớp định sẵn \\
2 & Object Detection & Xác định vị trí (bounding box) và nhãn của đối tượng trong ảnh \\
3 & Semantic Segmentation & Gán nhãn cho từng pixel trong ảnh \\
4 & Instance Segmentation & Phân biệt các instance khác nhau của cùng một lớp đối tượng \\
5 & Pose Estimation & Ước lượng tư thế (skeleton) của đối tượng \\
\bottomrule
\end{tabular}
\end{table}

Trong đề tài này, chúng tôi kết hợp hai bài toán: \textbf{Object Detection} (phát hiện vị trí chuối) và \textbf{Image Classification} (phân loại độ chín trên vùng crop).

\section{Mạng nơ-ron tích chập (CNN)}

\subsection{Nguyên lý hoạt động}

\Gls{cnn} là kiến trúc nền tảng cho hầu hết các mô hình thị giác máy tính hiện đại. Ý tưởng cốt lõi của CNN dựa trên hai nguyên lý:

\begin{enumerate}
    \item \textbf{Local connectivity}: Mỗi neuron chỉ kết nối với một vùng nhỏ của input (receptive field), phù hợp với đặc tính địa phương của ảnh.
    
    \item \textbf{Weight sharing}: Các tham số (kernel/filter) được chia sẻ trên toàn bộ ảnh, giảm số lượng tham số và tăng khả năng tổng quát hóa.
\end{enumerate}

\textbf{Tại sao CNN hiệu quả cho bài toán ảnh?}

Từ góc nhìn lý thuyết, sự thành công của CNN có thể được giải thích qua \textit{inductive bias}. Mỗi kiến trúc mạng nơ-ron ẩn chứa một số giả định về dữ liệu. CNN giả định rằng:
\begin{itemize}
    \item \textbf{Translation equivariance}: Nếu đối tượng dịch chuyển trong ảnh, feature tương ứng cũng dịch chuyển theo.
    \item \textbf{Locality}: Đặc trưng thị giác được xây dựng từ các pattern cục bộ (cạnh, góc, texture) trước khi tổng hợp thành pattern toàn cục.
    \item \textbf{Hierarchical compositionality}: Đặc trưng cấp cao được tạo từ đặc trưng cấp thấp (pixel $\rightarrow$ edge $\rightarrow$ texture $\rightarrow$ part $\rightarrow$ object).
\end{itemize}

Những giả định này phù hợp đặc biệt tốt với ảnh tự nhiên, bao gồm cả ảnh chuối trong đề tài của chúng tôi.

\begin{figure}[H]
\centering
\begin{tikzpicture}[scale=0.8]
    % Input
    \draw[fill=blue!20] (0,0) rectangle (2,2);
    \node at (1,-0.4) {Input};
    
    % Conv1
    \draw[fill=green!20] (3,0.2) rectangle (4.5,1.8);
    \node at (3.75,-0.4) {Conv};
    
    % Pool1
    \draw[fill=yellow!20] (5.5,0.3) rectangle (6.5,1.7);
    \node at (6,-0.4) {Pool};
    
    % Conv2
    \draw[fill=green!20] (7.5,0.4) rectangle (8.5,1.6);
    \node at (8,-0.4) {Conv};
    
    % FC
    \draw[fill=red!20] (9.5,0.5) rectangle (10.2,1.5);
    \node at (9.85,-0.4) {FC};
    
    % Output
    \draw[fill=purple!20] (11.2,0.7) rectangle (11.8,1.3);
    \node at (11.5,-0.4) {Out};
    
    % Arrows
    \draw[->, thick] (2,1) -- (3,1);
    \draw[->, thick] (4.5,1) -- (5.5,1);
    \draw[->, thick] (6.5,1) -- (7.5,1);
    \draw[->, thick] (8.5,1) -- (9.5,1);
    \draw[->, thick] (10.2,1) -- (11.2,1);
\end{tikzpicture}
\caption{Kiến trúc CNN cơ bản}
\label{fig:cnn_arch}
\end{figure}

\subsection{Phép tích chập (Convolution)}

Phép tích chập 2D giữa input $I$ và kernel $K$ kích thước $m \times n$ được định nghĩa:

\begin{equation}
(I * K)(i,j) = \sum_{p=0}^{m-1} \sum_{q=0}^{n-1} I(i+p, j+q) \cdot K(p,q)
\label{eq:convolution}
\end{equation}

Kernel ``trượt'' qua toàn bộ ảnh, tạo ra feature map mới. Các kernel khác nhau trích xuất các đặc trưng khác nhau: cạnh, góc, kết cấu, v.v.

\subsection{Các thành phần chính}

\begin{itemize}
    \item \textbf{Convolutional Layer}: Áp dụng nhiều filter để trích xuất đặc trưng.
    \item \textbf{Pooling Layer}: Giảm kích thước spatial, tăng tính bất biến với dịch chuyển. Phổ biến nhất là Max Pooling.
    \item \textbf{Activation Function}: ReLU, Leaky ReLU, SiLU (Swish) --- thêm tính phi tuyến.
    \item \textbf{Batch Normalization}: Chuẩn hóa output theo batch, ổn định quá trình training.
    \item \textbf{Fully Connected Layer}: Kết nối toàn phần ở cuối mạng để phân loại.
\end{itemize}

\section{Kiến trúc YOLO (You Only Look Once)}

\subsection{Lịch sử phát triển}

\Gls{yolo} được giới thiệu lần đầu bởi Redmon et al. (2016) với ý tưởng đột phá: coi object detection như một bài toán hồi quy đơn lẻ, dự đoán bounding box và class probability trong một lần forward pass. Đây là bước nhảy vọt so với các phương pháp 2-stage như R-CNN.

\begin{table}[H]
\centering
\caption{Tiến hóa của kiến trúc YOLO}
\label{tab:yolo_evolution}
\begin{tabular}{@{}lllp{5cm}@{}}
\toprule
\textbf{Version} & \textbf{Năm} & \textbf{Tác giả} & \textbf{Đặc điểm nổi bật} \\
\midrule
YOLOv1 & 2016 & Redmon et al. & Ý tưởng one-stage đầu tiên \\
YOLOv2 & 2017 & Redmon et al. & Batch Norm, anchor boxes \\
YOLOv3 & 2018 & Redmon et al. & Multi-scale detection \\
YOLOv4 & 2020 & Bochkovskiy et al. & CSPDarknet, PANet \\
YOLOv5 & 2020 & Ultralytics & PyTorch, model scaling \\
YOLOv8 & 2023 & Ultralytics & Anchor-free, SOTA accuracy \\
YOLO11 & 2024 & Ultralytics & Optimized efficiency \\
\bottomrule
\end{tabular}
\end{table}

\subsection{YOLOv8 --- Kiến trúc sử dụng trong đề tài}

YOLOv8 (Ultralytics, 2023) là phiên bản mới nhất tại thời điểm thực hiện đề tài, với các cải tiến chính:

\begin{itemize}
    \item \textbf{Anchor-free detection}: Loại bỏ anchor boxes, đơn giản hóa thiết kế.
    \item \textbf{C2f module}: Kết hợp CSP (Cross Stage Partial) với ELAN, tăng gradient flow.
    \item \textbf{Decoupled head}: Tách riêng classification và regression head.
    \item \textbf{Unified API}: Hỗ trợ detection, segmentation, classification, pose trong cùng framework.
\end{itemize}

\begin{figure}[H]
\centering
\begin{tikzpicture}[scale=0.7, every node/.style={scale=0.8}]
    % Backbone
    \draw[fill=blue!30, rounded corners] (0,0) rectangle (3,4);
    \node[rotate=90] at (1.5,2) {\textbf{Backbone}};
    \node at (1.5,3.5) {CSPDarknet};
    \node at (1.5,0.5) {C2f blocks};
    
    % Neck
    \draw[fill=green!30, rounded corners] (4,0) rectangle (7,4);
    \node[rotate=90] at (5.5,2) {\textbf{Neck}};
    \node at (5.5,3.5) {PANet};
    \node at (5.5,0.5) {FPN};
    
    % Head
    \draw[fill=red!30, rounded corners] (8,0) rectangle (11,4);
    \node[rotate=90] at (9.5,2) {\textbf{Head}};
    \node at (9.5,3.5) {Decoupled};
    \node at (9.5,0.5) {Cls + Reg};
    
    % Arrows
    \draw[->, thick] (3,2) -- (4,2);
    \draw[->, thick] (7,2) -- (8,2);
    \draw[->, thick] (11,2) -- (12,2);
    \node at (12.5,2) {Output};
\end{tikzpicture}
\caption{Kiến trúc tổng quan YOLOv8}
\label{fig:yolov8_arch}
\end{figure}

\subsection{YOLOv8 Classification Mode}

Ngoài detection, YOLOv8 còn hỗ trợ \textbf{classification mode} (yolov8n-cls.pt), trong đó:
\begin{itemize}
    \item Backbone trích xuất đặc trưng từ toàn bộ ảnh.
    \item Global Average Pooling tổng hợp feature map.
    \item Fully Connected Layer dự đoán xác suất từng class.
\end{itemize}

Đây là mode được sử dụng cho classifier trong pipeline của đề tài.

\section{Không gian màu và phân tích màu sắc}

\subsection{Không gian màu RGB}

RGB (Red-Green-Blue) là không gian màu phổ biến nhất trong hình ảnh số, với mỗi pixel được biểu diễn bởi 3 thành phần:

\begin{equation}
\text{Pixel} = (R, G, B), \quad R, G, B \in [0, 255]
\end{equation}

Tuy nhiên, RGB không phù hợp cho phân tích màu sắc vì:
\begin{itemize}
    \item Các kênh có tương quan cao, khó tách riêng thông tin màu.
    \item Nhạy cảm với điều kiện chiếu sáng.
\end{itemize}

\subsection{Không gian màu HSV}

\Gls{hsv} tách riêng thông tin màu sắc (Hue), độ bão hòa (Saturation) và độ sáng (Value):

\begin{equation}
\text{HSV} = (H, S, V), \quad H \in [0, 180], \; S, V \in [0, 255]
\end{equation}

Trong đề tài, chúng tôi sử dụng HSV để định nghĩa các ngưỡng màu cho chuối:

\begin{table}[H]
\centering
\caption{Ngưỡng HSV cho phân loại màu chuối}
\label{tab:hsv_thresholds}
\begin{tabular}{@{}lcc@{}}
\toprule
\textbf{Màu sắc} & \textbf{HSV thấp} & \textbf{HSV cao} \\
\midrule
Vàng (chín) & (15, 80, 80) & (35, 255, 255) \\
Xanh (xanh) & (35, 40, 40) & (85, 255, 255) \\
Nâu (quá chín) & (5, 50, 20) & (20, 200, 150) \\
Đen (hỏng) & (0, 0, 0) & (180, 255, 50) \\
\bottomrule
\end{tabular}
\end{table}

\subsection{Không gian màu LAB}

\Gls{lab} (CIE L*a*b*) được thiết kế để gần với nhận thức màu của mắt người:
\begin{itemize}
    \item $L^*$: Lightness (độ sáng), [0, 100]
    \item $a^*$: Green-Red axis
    \item $b^*$: Blue-Yellow axis
\end{itemize}

LAB được sử dụng trong đề tài cho:
\begin{itemize}
    \item \textbf{\Gls{clahe}}: Tăng cường độ tương phản trên kênh L.
    \item \textbf{Color uniformity}: Tính độ đồng đều màu sắc qua độ lệch chuẩn.
\end{itemize}

\section{Các kỹ thuật xử lý ảnh}

\subsection{Morphological Operations}

Các phép toán hình thái học thao tác trên hình dạng của đối tượng trong ảnh nhị phân:

\begin{itemize}
    \item \textbf{Erosion}: Co nhỏ đối tượng, loại bỏ noise nhỏ.
    \item \textbf{Dilation}: Mở rộng đối tượng, lấp đầy lỗ hổng.
    \item \textbf{Opening}: Erosion $\rightarrow$ Dilation, loại noise.
    \item \textbf{Closing}: Dilation $\rightarrow$ Erosion, lấp lỗ.
\end{itemize}

\subsection{Contour Analysis}

Contour (đường viền) của đối tượng cung cấp thông tin hình thái quan trọng:

\begin{equation}
\text{Solidity} = \frac{\text{Area}}{\text{Convex Hull Area}}
\end{equation}

\begin{equation}
\text{Aspect Ratio} = \frac{\text{Width}}{\text{Height}}
\end{equation}

Các đặc trưng này giúp phân biệt chuối lành (solidity cao, hình dạng đều) với chuối hỏng (bề mặt không đều).

\subsection{Texture Analysis}

Kết cấu bề mặt được đo bằng \textbf{Laplacian variance}:

\begin{equation}
\text{Texture Variance} = \text{Var}(\nabla^2 I)
\end{equation}

trong đó $\nabla^2 I$ là Laplacian của ảnh grayscale. Giá trị cao chỉ ra bề mặt gồ ghề (có thể do đốm, vết thối).

\section{Đánh giá mô hình}

\subsection{Confusion Matrix}

Ma trận nhầm lẫn (Confusion Matrix) là công cụ cơ bản để đánh giá mô hình phân loại. Nó cung cấp cái nhìn chi tiết về số lượng mẫu được phân loại đúng/sai cho từng class, giúp xác định các \textit{kiểu lỗi} của mô hình. Với bài toán 4 class (unripe/ripe/overripe/rotten), confusion matrix 4$\times$4 cho phép phân tích xem mô hình có học được ``thứ tự chín'' hay không --- nếu các nhầm lẫn chỉ xảy ra giữa các class liền kề, đó là dấu hiệu tích cực.

Với nền tảng lý thuyết này, chương tiếp theo sẽ trình bày \textbf{phương pháp thực hiện} --- cách chúng tôi thiết kế hệ thống dựa trên các khái niệm đã giới thiệu.

\clearpage

\begin{table}[H]
\centering
\caption{Cấu trúc Confusion Matrix cho 2 lớp}
\label{tab:confusion_matrix}
\begin{tabular}{@{}l|cc@{}}
\toprule
& \textbf{Predicted Positive} & \textbf{Predicted Negative} \\
\midrule
\textbf{Actual Positive} & True Positive (TP) & False Negative (FN) \\
\textbf{Actual Negative} & False Positive (FP) & True Negative (TN) \\
\bottomrule
\end{tabular}
\end{table}

\subsection{Các chỉ số đánh giá}

\begin{itemize}
    \item \textbf{Accuracy}:
    \begin{equation}
    \text{Accuracy} = \frac{TP + TN}{TP + TN + FP + FN}
    \end{equation}
    
    \item \textbf{Precision} (độ chính xác):
    \begin{equation}
    \text{Precision} = \frac{TP}{TP + FP}
    \end{equation}
    
    \item \textbf{Recall} (độ phủ):
    \begin{equation}
    \text{Recall} = \frac{TP}{TP + FN}
    \end{equation}
    
    \item \textbf{F1-Score}:
    \begin{equation}
    F_1 = 2 \cdot \frac{\text{Precision} \cdot \text{Recall}}{\text{Precision} + \text{Recall}}
    \end{equation}
\end{itemize}

Trong bài toán phân loại chuối hỏng, \textbf{Recall cho class ``defective'' đặc biệt quan trọng}: chúng ta muốn phát hiện được tất cả chuối hỏng (ít false negative), thà loại nhầm còn hơn bỏ sót.

\subsection{Top-1 và Top-5 Accuracy}

Trong classification, ngoài accuracy tiêu chuẩn, ta còn đánh giá:
\begin{itemize}
    \item \textbf{Top-1 Accuracy}: Tỉ lệ mẫu mà class dự đoán cao nhất đúng.
    \item \textbf{Top-5 Accuracy}: Tỉ lệ mẫu mà class đúng nằm trong top 5 dự đoán.
\end{itemize}

\section{Tóm tắt chương}

Chương này đã trình bày nền tảng lý thuyết cho hệ thống phân loại chất lượng chuối:
\begin{itemize}
    \item \textbf{Thị giác máy tính} và các bài toán cơ bản (detection, classification).
    \item \textbf{Mạng nơ-ron tích chập (CNN)} với các nguyên lý: local connectivity, weight sharing, hierarchical compositionality.
    \item \textbf{Kiến trúc YOLO} và sự tiến hóa từ v1 đến v8, đặc biệt là YOLOv8 classification mode.
    \item \textbf{Không gian màu HSV/LAB} và ứng dụng trong phân tích màu sắc chuối.
    \item \textbf{Các metrics đánh giá}: Confusion Matrix, Accuracy, Precision, Recall, F1-Score.
\end{itemize}

Với nền tảng lý thuyết này, chương tiếp theo sẽ trình bày \textbf{phương pháp thực hiện} --- cách chúng tôi thiết kế hệ thống 2-stage pipeline dựa trên các khái niệm đã giới thiệu.

\clearpage

% ==============================================================================
% CHƯƠNG 3: PHƯƠNG PHÁP THỰC HIỆN
% ==============================================================================

\chapter{PHƯƠNG PHÁP THỰC HIỆN}
\label{chap:methodology}

Chương này trình bày chi tiết phương pháp tiếp cận và thiết kế hệ thống. Thay vì chỉ mô tả \textit{làm gì}, chúng tôi cố gắng trả lời \textit{tại sao} --- giải thích lý do đằng sau mỗi quyết định thiết kế, từ góc nhìn của một nhà nghiên cứu muốn hiểu sâu bản chất vấn đề.

\section{Tổng quan kiến trúc hệ thống}

\subsection{Triết lý thiết kế}

Hệ thống phân loại chất lượng chuối được thiết kế theo kiến trúc \textbf{2 giai đoạn} (two-stage pipeline), kết hợp điểm mạnh của object detection và image classification. Lựa chọn này không chỉ xuất phát từ yêu cầu kỹ thuật mà còn từ nhận thức sâu sắc về bản chất dữ liệu: hầu hết các tập dữ liệu công khai về độ chín chuối đều là classification datasets (folder-per-class, không có bounding box), trong khi giao diện người dùng lại yêu cầu hiển thị vị trí của chuối trong khung hình.

\textbf{Quyết định thiết kế cốt lõi:}

\begin{enumerate}
    \item \textbf{Constraints dẫn dắt thiết kế, không phải ngược lại}: Việc không có bounding box annotation trong Kaggle dataset là một constraint, nhưng thay vì coi đó là hạn chế, chúng tôi biến nó thành cơ hội để tận dụng COCO pretrained detector --- một mô hình đã được huấn luyện trên hàng triệu ảnh.
    
    \item \textbf{Separation of concerns}: Detector lo việc \textit{đâu là chuối}, classifier lo việc \textit{chuối này chất lượng thế nào}. Mỗi component có thể được cải tiến độc lập mà không ảnh hưởng đến component khác.
    
    \item \textbf{Transfer learning như là lợi thế cạnh tranh}: Với nguồn lực hạn chế của một dự án học thuật, việc kế thừa tri thức từ các mô hình pretrained là chiến lược thông minh hơn việc cố gắng train from scratch.
\end{enumerate}

\begin{figure}[H]
\centering
\begin{tikzpicture}[
    node distance=1.5cm,
    block/.style={rectangle, draw, fill=blue!20, text width=3cm, text centered, rounded corners, minimum height=1.2cm},
    decision/.style={diamond, draw, fill=yellow!20, text width=2.5cm, text centered, aspect=2},
    arrow/.style={->, thick}
]
    % Input
    \node[block, fill=green!30] (input) {Video Frame\\(Webcam)};
    
    % Detector
    \node[block, below of=input] (detector) {YOLO Detector\\(COCO pretrained)};
    
    % Decision
    \node[decision, below of=detector, yshift=-0.5cm] (check) {Tìm thấy\\chuối?};
    
    % Classifier
    \node[block, below of=check, yshift=-0.5cm] (classifier) {YOLO Classifier\\(Fine-tuned)};
    
    % Analyzer
    \node[block, right of=classifier, xshift=3cm] (analyzer) {Banana Analyzer\\(Feature Extraction)};
    
    % Aggregator
    \node[block, below of=classifier] (aggregator) {Result Aggregator\\(Multi-bbox)};
    
    % Output
    \node[block, below of=aggregator, fill=red!30] (output) {UI Display\\(Grading Result)};
    
    % No detection path
    \node[block, right of=check, xshift=3cm, fill=gray!30] (nodetect) {Không phát hiện\\(Hold bbox)};
    
    % Arrows
    \draw[arrow] (input) -- (detector);
    \draw[arrow] (detector) -- (check);
    \draw[arrow] (check) -- node[left] {Có} (classifier);
    \draw[arrow] (check) -- node[above] {Không} (nodetect);
    \draw[arrow] (nodetect) |- (aggregator);
    \draw[arrow] (classifier) -- (analyzer);
    \draw[arrow] (classifier) -- (aggregator);
    \draw[arrow] (analyzer) -- (aggregator);
    \draw[arrow] (aggregator) -- (output);
\end{tikzpicture}
\caption{Kiến trúc tổng quan hệ thống phân loại chất lượng chuối}
\label{fig:system_architecture}
\end{figure}

\section{Pipeline xử lý 2 giai đoạn}

\subsection{Giai đoạn 1: Object Detection}

Mục tiêu của giai đoạn này là xác định \textbf{vị trí} của tất cả quả chuối trong khung hình. Chúng tôi sử dụng YOLOv8n pretrained trên COCO dataset, trong đó class ``banana'' đã được định nghĩa sẵn (class ID = 46 trong COCO).

\textbf{Lý do không train detector riêng từ đầu:}
\begin{enumerate}
    \item \textbf{COCO detector đã đủ tốt cho việc định vị chuối trong điều kiện bình thường}. COCO dataset bao gồm 80 class, trong đó ``banana'' (class ID 46) đã được train trên hàng ngàn ảnh đa dạng.
    
    \item \textbf{Tiết kiệm công sức label bounding box} --- một công việc tốn thời gian và dễ sai sót. Với dataset 13,500 ảnh, việc vẽ bounding box thủ công sẽ tiêu tốn vài tuần làm việc.
    
    \item \textbf{Tập trung nguồn lực vào classifier} --- nơi thực sự quyết định chất lượng phân loại. Đây là nguyên tắc \textit{``invest where it matters''} trong thiết kế hệ thống.
\end{enumerate}

\textbf{Phân tích trade-off:}

Quyết định sử dụng COCO pretrained detector có trade-off rõ ràng:
\begin{itemize}
    \item \textbf{Pros}: Nhanh chóng deploy, không cần annotation, tận dụng tri thức từ dataset lớn.
    \item \textbf{Cons}: Có thể miss detection trong một số trường hợp (góc chụp lạ, ánh sáng yếu, chuối bị che một phần).
\end{itemize}

Chúng tôi giải quyết cons bằng cơ chế \textbf{Temporal Stabilization (Bbox Hold)} được trình bày ở phần sau.

\textbf{Xử lý đa đối tượng:}
Khi phát hiện nhiều quả chuối trong cùng một frame, hệ thống:
\begin{itemize}
    \item Sắp xếp theo diện tích bounding box giảm dần.
    \item Giới hạn số quả xử lý bằng tham số \texttt{BANANA\_MAX\_FRUITS} (mặc định 6).
    \item Xử lý song song nếu tài nguyên cho phép.
\end{itemize}

\subsection{Giai đoạn 2: Classification}

Với mỗi bounding box, hệ thống cắt (crop) vùng ảnh tương ứng và đưa vào classifier. Classifier là mô hình YOLOv8n-cls được fine-tune trên tập dữ liệu Kaggle Banana Ripeness.

\begin{equation}
P(y | \mathbf{x}_{\text{crop}}) = \text{Softmax}(f_\theta(\mathbf{x}_{\text{crop}}))
\end{equation}

trong đó $f_\theta$ là backbone + classifier head, $\mathbf{x}_{\text{crop}}$ là ảnh crop từ bounding box.

\subsection{Temporal Stabilization (Bbox Hold)}

Một vấn đề thực tế khi sử dụng COCO detector là sự \textbf{nhấp nháy} (flickering) --- detector có thể bỏ sót chuối ở một vài frame do góc chụp hoặc motion blur tạm thời. Để giải quyết:

\begin{itemize}
    \item Lưu trữ bounding box gần nhất thành công.
    \item Nếu detector không tìm thấy chuối trong frame hiện tại, sử dụng bbox đã lưu trong tối đa \texttt{BANANA\_BBOX\_HOLD} frame (mặc định 5).
    \item Đánh dấu kết quả là ``bbox\_held'' để UI có thể hiển thị khác biệt nếu cần.
\end{itemize}

\section{Module BananaAnalyzer}

Đây là module phân tích hình ảnh nâng cao, được phát triển dựa trên các nghiên cứu khoa học về nhận diện chuối \ref{ref:banana_detection_paper}. Module này hoạt động như một ``lớp refinement'' cho kết quả của classifier.

\subsection{Tiền xử lý ảnh}

\begin{lstlisting}[caption={Tiền xử lý ảnh trong BananaAnalyzer},label={lst:preprocess}]
def preprocess_frame(self, frame_bgr, enhance_contrast=True, denoise=True):
    result = frame_bgr.copy()
    
    # Bilateral filter: loai noise ma giu canh
    if denoise:
        result = cv2.bilateralFilter(result, d=5, sigmaColor=50, sigmaSpace=50)
    
    # CLAHE trên kênh L của LAB
    if enhance_contrast:
        lab = cv2.cvtColor(result, cv2.COLOR_BGR2LAB)
        l_channel, a, b = cv2.split(lab)
        clahe = cv2.createCLAHE(clipLimit=2.0, tileGridSize=(8,8))
        l_enhanced = clahe.apply(l_channel)
        lab_enhanced = cv2.merge([l_enhanced, a, b])
        result = cv2.cvtColor(lab_enhanced, cv2.COLOR_LAB2BGR)
    
    return result
\end{lstlisting}

\subsection{Trích xuất đặc trưng màu sắc}

Từ không gian HSV, chúng tôi tính các tỉ lệ màu sắc đặc trưng:

\begin{table}[H]
\centering
\caption{Các đặc trưng màu sắc được trích xuất}
\label{tab:color_features}
\begin{tabular}{@{}lp{8cm}@{}}
\toprule
\textbf{Đặc trưng} & \textbf{Ý nghĩa} \\
\midrule
yellow\_ratio & Tỉ lệ pixel vàng, chỉ báo độ chín \\
green\_ratio & Tỉ lệ pixel xanh, chỉ báo chưa chín \\
brown\_ratio & Tỉ lệ pixel nâu, chỉ báo quá chín \\
black\_ratio & Tỉ lệ pixel đen, chỉ báo hỏng/thối \\
color\_uniformity & Độ đồng đều màu (1 - normalized std của LAB) \\
\bottomrule
\end{tabular}
\end{table}

\subsection{Phát hiện đốm và khuyết tật}

\begin{equation}
\text{spot\_ratio} = \frac{\sum_{i} \mathbf{1}[\text{Area}(C_i) > \text{min\_spot\_area}]}{\text{Total Area}}
\end{equation}

trong đó $C_i$ là các contour đốm được phát hiện qua adaptive thresholding kết hợp với mask màu nâu/đen.

\subsection{Tính điểm chất lượng tổng hợp}

Điểm chất lượng (\texttt{quality\_score}) được tính từ các đặc trưng theo công thức:

\begin{equation}
Q = w_1 \cdot \text{yellow\_ratio} - w_2 \cdot \text{brown\_ratio} - w_3 \cdot \text{black\_ratio} + w_4 \cdot \text{color\_uniformity} - w_5 \cdot \text{spot\_ratio}
\end{equation}

với các trọng số $w_i$ được điều chỉnh theo kinh nghiệm thực tế.

\section{Phân loại 4 cấp độ}

Hệ thống ánh xạ kết quả classifier sang 4 cấp độ hành động:

\begin{table}[H]
\centering
\caption{Bảng ánh xạ class sang category}
\label{tab:class_mapping}
\begin{tabular}{@{}clll@{}}
\toprule
\textbf{Class ID} & \textbf{Tên class} & \textbf{Category} & \textbf{Hành động} \\
\midrule
0 & fresh/green & unripe & Chưa thu hoạch \\
1 & ripe/yellow & export & Bảo quản \& Ship hàng \\
2 & overripe & overripe & Cần bán/ăn ngay \\
3 & rotten & defective & Loại bỏ \\
\bottomrule
\end{tabular}
\end{table}

\textbf{Cơ chế auto-mapping:} Nếu có file \texttt{data.yaml}, hệ thống tự động suy luận mapping từ tên class:

\begin{lstlisting}[caption={Suy luận category từ tên class},label={lst:infer_category}]
def _infer_category_from_class_name(self, name):
    tokens = set(name.lower().split())
    
    if tokens.intersection({'rotten', 'mold', 'defect', 'spoiled'}):
        return "defective"
    if tokens.intersection({'overripe', 'brown', 'spotted'}):
        return "overripe"
    if tokens.intersection({'unripe', 'green', 'fresh'}):
        return "unripe"
    if tokens.intersection({'ripe', 'yellow', 'mature'}):
        return "export"
    
    return None
\end{lstlisting}

\section{Feature Refinement}

Khi classifier trả về kết quả có độ tin cậy thấp (uncertain), hoặc khi phát hiện tín hiệu defective từ analyzer, hệ thống áp dụng \textbf{feature refinement}:

\begin{algorithm}[H]
\caption{Feature Refinement Algorithm}
\label{alg:refinement}
\begin{algorithmic}[1]
\Require $\text{cls\_result}$: kết quả classifier, $\text{features}$: đặc trưng từ analyzer
\Ensure $\text{final\_category}$: category sau refinement

\If{$\text{features.black\_ratio} > 0.15$}
    \State \Return \texttt{"defective"}
\EndIf

\If{$\text{features.brown\_ratio} > 0.3$ \textbf{and} $\text{features.spot\_count} > 10$}
    \State \Return \texttt{"defective"}
\EndIf

\If{$\text{cls\_result.confidence} < 0.5$}
    \State Sử dụng color\_ratio để quyết định
\Else
    \State \Return $\text{cls\_result.category}$
\EndIf
\end{algorithmic}
\end{algorithm}

\section{Tổng hợp kết quả đa bbox}

Khi frame có nhiều quả chuối, hệ thống tổng hợp kết quả theo nguyên tắc \textbf{severity ranking}:

\begin{equation}
\text{overall} = \arg\max_{r \in \text{items}} \left( \text{severity}(r), \text{confidence}(r) \right)
\end{equation}

với thứ tự severity: defective > overripe > export > unripe > none.

\textbf{Ý nghĩa:} Nếu trong nải có bất kỳ quả hỏng nào, toàn bộ nải được đánh dấu là defective --- đây là nguyên tắc an toàn trong kiểm soát chất lượng thực phẩm.

\section{Xử lý video và threading}

Để đảm bảo giao diện mượt mà, hệ thống tách biệt các luồng xử lý:

\begin{figure}[H]
\centering
\begin{tikzpicture}[
    thread/.style={rectangle, draw, fill=blue!20, minimum width=3cm, minimum height=1cm, rounded corners},
    queue/.style={rectangle, draw, fill=yellow!20, minimum width=2cm, minimum height=0.6cm},
    arrow/.style={->, thick}
]
    % Capture thread
    \node[thread] (capture) at (0,2) {Capture Thread};
    \node[below of=capture, node distance=0.8cm, font=\small] {30 FPS};
    
    % Queue
    \node[queue] (queue) at (4,2) {Frame Queue};
    
    % Grade thread
    \node[thread] (grade) at (8,2) {Grade Thread};
    \node[below of=grade, node distance=0.8cm, font=\small] {Inference};
    
    % UI thread
    \node[thread, fill=green!30] (ui) at (4,0) {UI Thread};
    \node[below of=ui, node distance=0.8cm, font=\small] {Display};
    
    % Arrows
    \draw[arrow] (capture) -- (queue);
    \draw[arrow] (queue) -- (grade);
    \draw[arrow] (grade) -- (ui);
    \draw[arrow] (capture) to[bend right=30] (ui);
\end{tikzpicture}
\caption{Kiến trúc đa luồng của hệ thống}
\label{fig:threading}
\end{figure}

\begin{itemize}
    \item \textbf{Capture Thread}: Đọc frame từ webcam với tốc độ cao nhất có thể.
    \item \textbf{Grade Thread}: Chạy inference detector + classifier + analyzer.
    \item \textbf{UI Thread}: Hiển thị kết quả, không bị block bởi inference.
    \item \textbf{Frame Queue}: Buffer 1 frame, drop frame cũ nếu inference chậm.
\end{itemize}

\section{Các tham số tối ưu hóa}

Hệ thống cung cấp nhiều tham số điều chỉnh qua biến môi trường:

\begin{table}[H]
\centering
\caption{Các tham số tối ưu hóa hiệu năng}
\label{tab:perf_params}
\begin{tabular}{@{}llp{6cm}@{}}
\toprule
\textbf{Tham số} & \textbf{Mặc định} & \textbf{Mô tả} \\
\midrule
BANANA\_MAX\_FRUITS & 6 & Số quả tối đa xử lý mỗi frame \\
BANANA\_DET\_IMGSZ & 640 & Kích thước input detector \\
BANANA\_CLS\_IMGSZ & 416 & Kích thước input classifier \\
BANANA\_BBOX\_HOLD & 5 & Số frame giữ bbox khi miss \\
BANANA\_ANALYZE\_POLICY & all & Policy chạy analyzer \\
BANANA\_MAX\_INFER\_FPS & 0 & Giới hạn FPS inference (0 = không giới hạn) \\
\bottomrule
\end{tabular}
\end{table}

\section{Tóm tắt chương}

Chương này đã trình bày chi tiết phương pháp thiết kế hệ thống phân loại chất lượng chuối:
\begin{itemize}
    \item \textbf{Kiến trúc 2-stage}: Kết hợp COCO detector (localization) với fine-tuned classifier (classification), tận dụng thế mạnh của từng thành phần.
    \item \textbf{BananaAnalyzer}: Module trích xuất đặc trưng màu sắc (HSV/LAB), hình thái và kết cấu bề mặt.
    \item \textbf{Feature Refinement}: Cơ chế hiệu chỉnh kết quả cho edge cases.
    \item \textbf{Temporal Stabilization}: Xử lý nhấp nháy bbox qua cơ chế hold.
\end{itemize}

\textbf{Từ thiết kế đến hiện thực:} Chương tiếp theo sẽ trình bày quá trình \textbf{cài đặt và triển khai} --- biến các quyết định thiết kế thành mã nguồn hoạt động.

\clearpage

% ==============================================================================
% CHƯƠNG 4: CÀI ĐẶT VÀ TRIỂN KHAI
% ==============================================================================

\chapter{CÀI ĐẶT VÀ TRIỂN KHAI}
\label{chap:implementation}

Chương này trình bày chi tiết quá trình cài đặt và triển khai hệ thống, từ việc chuẩn bị môi trường phát triển, thu thập dữ liệu huấn luyện, đến việc đóng gói sản phẩm cuối cùng. Các kết quả đánh giá chi tiết sẽ được trình bày ở Chương~\ref{chap:results}.

\section{Môi trường phát triển}

\subsection{Yêu cầu phần cứng}

\begin{table}[H]
\centering
\caption{Cấu hình phần cứng khuyến nghị}
\label{tab:hardware}
\begin{tabular}{@{}lll@{}}
\toprule
\textbf{Thành phần} & \textbf{Tối thiểu} & \textbf{Khuyến nghị} \\
\midrule
CPU & Intel Core i5 / AMD Ryzen 5 & Intel Core i7 / AMD Ryzen 7 \\
RAM & 8 GB & 16 GB \\
GPU & Tích hợp (iGPU) & NVIDIA GTX 1650 trở lên \\
Webcam & 720p & 1080p \\
Storage & 5 GB SSD & 10 GB SSD \\
\bottomrule
\end{tabular}
\end{table}

\subsection{Yêu cầu phần mềm}

\begin{table}[H]
\centering
\caption{Các thư viện Python sử dụng}
\label{tab:requirements}
\begin{tabular}{@{}llp{6cm}@{}}
\toprule
\textbf{Thư viện} & \textbf{Phiên bản} & \textbf{Mục đích} \\
\midrule
Python & $\geq$ 3.9 & Ngôn ngữ lập trình chính \\
opencv-python & $\geq$ 4.8.0 & Xử lý ảnh và video \\
numpy & $\geq$ 1.24.0 & Tính toán ma trận \\
pillow & $\geq$ 10.0.0 & Vẽ text Unicode \\
customtkinter & $\geq$ 5.2.0 & Giao diện người dùng \\
ultralytics & $\geq$ 8.0.0 & YOLO framework \\
torch & (tự động) & Backend deep learning \\
kaggle & $\geq$ 1.6.0 & Tải dataset \\
pytest & $\geq$ 7.0 & Unit testing \\
\bottomrule
\end{tabular}
\end{table}

\section{Cấu trúc dự án}

Mã nguồn được tổ chức theo cấu trúc module hóa rõ ràng:

\begin{lstlisting}[language=bash,caption={Cấu trúc thư mục dự án},label={lst:project_structure}]
banana-quality-grading/
|-- main.py                  # Entry point
|-- requirements.txt         # Dependencies
|-- training_script.py       # Train detector
|-- training_kaggle_classification.py  # Train classifier
|-- export_android_models.py # Export TFLite
|
|-- app/
|   |-- __init__.py
|   |-- banana_analyzer.py   # Feature extraction module
|   |-- grader.py            # Main grading logic
|   |-- text_overlay.py      # Vietnamese text rendering
|   |-- ui_manager.py        # CustomTkinter UI
|   |-- video_thread.py      # Capture & threading
|
|-- utils/
|   |-- resource_manager.py  # Auto-download resources
|
|-- weights/
|   |-- best.pt              # Classifier weights
|   |-- detector.pt          # Custom detector (optional)
|
|-- datasets/
|   |-- kaggle_banana_ripeness/  # Kaggle dataset
|
|-- runs_banana/             # Training outputs
|   |-- yolov8n_banana_cls/
|       |-- weights/best.pt
|
|-- tests/                   # Unit tests
    |-- test_grader_real_model_integration.py
\end{lstlisting}

\section{Huấn luyện mô hình}

\subsection{Dataset}

\subsubsection{Nguồn gốc và đặc điểm dữ liệu}

Chúng tôi sử dụng tập dữ liệu \textbf{Banana Classification Dataset} từ Kaggle \ref{ref:kaggle_banana_classification}, một dataset được xây dựng với mục đích huấn luyện các mô hình học máy phát hiện và phân loại chất lượng chuối. Điểm đặc biệt của dataset này là nó được tổng hợp từ dự án \textbf{Banana Ripeness Classification} trên nền tảng Roboflow Universe \ref{ref:roboflow_banana}, đảm bảo tính đa dạng và chất lượng annotation.

\begin{table}[H]
\centering
\caption{Thống kê chi tiết Banana Classification Dataset}
\label{tab:dataset_stats}
\begin{tabular}{@{}llr@{}}
\toprule
\textbf{Thuộc tính} & \textbf{Mô tả} & \textbf{Giá trị} \\
\midrule
\multicolumn{3}{@{}l}{\textit{Tổng quan}} \\
\midrule
Tổng số ảnh & Toàn bộ dataset & 13,500 ảnh \\
Kích thước dataset & Dung lượng lưu trữ & 227.18 MB \\
Giấy phép & Open source & MIT License \\
\midrule
\multicolumn{3}{@{}l}{\textit{Phân chia dữ liệu}} \\
\midrule
Training set & Dùng để huấn luyện & $\sim$10,800 ảnh (80\%) \\
Validation set & Đánh giá trong quá trình train & $\sim$1,350 ảnh (10\%) \\
Test set & Đánh giá cuối cùng & $\sim$1,350 ảnh (10\%) \\
\midrule
\multicolumn{3}{@{}l}{\textit{Phân bố các lớp}} \\
\midrule
Unripe (Xanh) & Chuối còn xanh, chưa chín & $\sim$3,375 ảnh \\
Ripe (Chín) & Chuối chín vàng, chất lượng tốt & $\sim$3,375 ảnh \\
Overripe (Quá chín) & Chuối có đốm nâu, bắt đầu mềm & $\sim$3,375 ảnh \\
Rotten (Thối) & Chuối hỏng, không sử dụng được & $\sim$3,375 ảnh \\
\bottomrule
\end{tabular}
\end{table}

\subsubsection{Phân tích chất lượng dữ liệu}

Từ góc nhìn của một nhà nghiên cứu, việc lựa chọn dataset không chỉ dừng lại ở số lượng mà còn phải xem xét kỹ lưỡng về \textit{chất lượng} và \textit{tính đại diện} của dữ liệu. Dataset này có một số điểm mạnh và hạn chế đáng lưu ý:

\textbf{Điểm mạnh:}
\begin{itemize}
    \item \textbf{Cân bằng giữa các lớp}: Phân bố đều giữa 4 class giúp tránh bias trong quá trình training.
    \item \textbf{Đa dạng điều kiện chụp}: Ảnh được thu thập từ nhiều nguồn, với các điều kiện ánh sáng và góc độ khác nhau.
    \item \textbf{Annotation rõ ràng}: Ranh giới giữa các class được định nghĩa tương đối rõ ràng dựa trên đặc điểm thị giác.
\end{itemize}

\textbf{Hạn chế cần lưu ý:}
\begin{itemize}
    \item \textbf{Không có bounding box annotation}: Dataset chỉ hỗ trợ classification, không có object detection labels. Đây chính là constraint cơ bản dẫn đến quyết định thiết kế pipeline 2 giai đoạn.
    \item \textbf{Nền ảnh đơn giản}: Phần lớn ảnh có nền trắng hoặc đơn sắc, có thể ảnh hưởng đến khả năng tổng quát hóa khi deploy trong môi trường thực tế.
    \item \textbf{Chủ yếu chuối Cavendish}: Dataset tập trung vào giống chuối tiêu, có thể không tổng quát hóa tốt sang các giống chuối địa phương Việt Nam.
\end{itemize}

\subsubsection{Lý do lựa chọn dataset}

Việc lựa chọn Banana Classification Dataset từ Kaggle xuất phát từ những suy xét mang tính phương pháp luận:

\begin{enumerate}
    \item \textbf{Tính mở và tái tạo được}: Dataset có giấy phép MIT, cho phép sử dụng cho mục đích học thuật và thương mại.
    
    \item \textbf{Kích thước phù hợp}: $\sim$13,500 ảnh đủ lớn để fine-tune một classifier từ pretrained weights, nhưng không quá lớn gây khó khăn về tài nguyên tính toán.
    
    \item \textbf{Cấu trúc chuẩn}: Định dạng folder-per-class tương thích trực tiếp với Ultralytics YOLO classification pipeline, giảm thiểu công sức tiền xử lý.
    
    \item \textbf{Cộng đồng hỗ trợ}: Dataset có hơn 2,000 lượt tải và nhiều notebooks tham khảo trên Kaggle, cho phép benchmark và so sánh kết quả.
\end{enumerate}

\subsection{Quy trình huấn luyện Classifier}

\begin{lstlisting}[language=bash,caption={Lệnh huấn luyện classifier},label={lst:train_cmd}]
# Buoc 1: Cau hinh Kaggle API
# Dat kaggle.json tai: %USERPROFILE%/.kaggle/kaggle.json

# Buoc 2: Chay script train (tu dong download dataset)
python training_kaggle_classification.py \
    --device auto \
    --epochs 50 \
    --imgsz 416 \
    --batch -1

# Buoc 3: Copy weights
copy runs_banana\yolov8n_banana_cls\weights\best.pt weights\best.pt
\end{lstlisting}

\subsection{Cấu hình huấn luyện}

\begin{table}[H]
\centering
\caption{Tham số huấn luyện classifier}
\label{tab:train_config}
\begin{tabular}{@{}ll@{}}
\toprule
\textbf{Tham số} & \textbf{Giá trị} \\
\midrule
Base model & yolov8n-cls.pt (pretrained ImageNet) \\
Image size & 416 $\times$ 416 \\
Epochs & 50 \\
Batch size & Auto (-1) \\
Optimizer & Auto (AdamW) \\
Learning rate & Auto-tuned \\
Device & GPU (CUDA) nếu có, ngược lại CPU \\
Augmentation & Default Ultralytics (flip, scale, HSV) \\
\bottomrule
\end{tabular}
\end{table}

\subsection{Kết quả huấn luyện}

Dựa trên file \texttt{results.csv} từ run \texttt{yolov8n\_banana\_cls3}:

\begin{table}[H]
\centering
\caption{Kết quả huấn luyện qua các epoch}
\label{tab:train_results}
\begin{tabular}{@{}ccccc@{}}
\toprule
\textbf{Epoch} & \textbf{Train Loss} & \textbf{Val Loss} & \textbf{Top-1 Acc} & \textbf{Top-5 Acc} \\
\midrule
1 & 0.6867 & 0.1191 & 96.35\% & 100\% \\
10 & 0.0537 & 0.0933 & 97.33\% & 100\% \\
20 & 0.0326 & 0.0623 & 98.66\% & 100\% \\
30 & 0.0218 & 0.1002 & 98.40\% & 100\% \\
40 & 0.0173 & 0.0638 & 98.66\% & 100\% \\
50 & 0.0096 & 0.0752 & 98.75\% & 100\% \\
\bottomrule
\end{tabular}
\end{table}

\begin{figure}[H]
\centering
\begin{tikzpicture}
\begin{axis}[
    xlabel={Epoch},
    ylabel={Accuracy (\%)},
    xmin=0, xmax=55,
    ymin=85, ymax=100,
    legend pos=south east,
    grid=major,
    width=12cm,
    height=7cm
]
\addplot[blue, thick, mark=*] coordinates {
    (1, 96.35) (5, 96.79) (10, 97.33) (15, 98.66) (20, 98.66)
    (25, 98.66) (30, 98.40) (35, 98.58) (40, 98.66) (45, 98.66) (50, 98.75)
};
\addlegendentry{Top-1 Accuracy}

\addplot[red, thick, mark=square] coordinates {
    (1, 100) (5, 100) (10, 100) (15, 100) (20, 100)
    (25, 100) (30, 100) (35, 100) (40, 100) (45, 100) (50, 100)
};
\addlegendentry{Top-5 Accuracy}
\end{axis}
\end{tikzpicture}
\caption{Biểu đồ accuracy qua quá trình huấn luyện}
\label{fig:train_curve}
\end{figure}

\textbf{Nhận xét sơ bộ:} Model hội tụ nhanh, đạt $>97\%$ Top-1 accuracy sau 10 epoch. Phân tích chi tiết về learning curve, confusion matrix và error pattern sẽ được trình bày ở Chương~\ref{chap:results}.

\begin{figure}[H]
\centering
\begin{subfigure}[b]{0.32\textwidth}
    \includegraphics[width=\textwidth]{images/train_batch0.jpg}
    \caption{Training batch 0}
\end{subfigure}
\hfill
\begin{subfigure}[b]{0.32\textwidth}
    \includegraphics[width=\textwidth]{images/train_batch1.jpg}
    \caption{Training batch 1}
\end{subfigure}
\hfill
\begin{subfigure}[b]{0.32\textwidth}
    \includegraphics[width=\textwidth]{images/train_batch2.jpg}
    \caption{Training batch 2}
\end{subfigure}
\caption{Một số batch dữ liệu huấn luyện đã qua augmentation. Các ảnh được tự động scale, flip và điều chỉnh màu sắc bởi Ultralytics pipeline, giúp tăng tính đa dạng và khả năng tổng quát hóa của mô hình.}
\label{fig:train_batches}
\end{figure}

\section{Phát triển giao diện người dùng}

\subsection{Thiết kế UI}

Giao diện được xây dựng bằng \textbf{CustomTkinter} --- thư viện mở rộng Tkinter với giao diện dark mode hiện đại.

\begin{figure}[H]
\centering
\includegraphics[width=0.9\textwidth]{images/screen.PNG}
\caption{Giao diện chính của ứng dụng}
\label{fig:ui_screenshot}
\end{figure}

\subsection{Các thành phần giao diện}

\begin{table}[H]
\centering
\caption{Mô tả các thành phần UI}
\label{tab:ui_components}
\begin{tabular}{@{}lp{8cm}@{}}
\toprule
\textbf{Thành phần} & \textbf{Chức năng} \\
\midrule
Video Panel & Hiển thị luồng webcam với bbox overlay và label \\
Status Label & Trạng thái hiện tại: ``Đang quay'', ``Đã dừng'', lỗi \\
Grade Label & Kết quả phân loại: ``Chuối Xanh'', ``Chín Vừa'', v.v. \\
Confidence Label & Độ tin cậy của classifier (\%) \\
FPS Label & Tốc độ xử lý hiện tại \\
Quality Panel & Thông tin chi tiết: điểm chất lượng, màu sắc, số đốm \\
Start/Stop Button & Bật/tắt camera \\
\bottomrule
\end{tabular}
\end{table}

\subsection{Hỗ trợ tiếng Việt}

OpenCV \texttt{cv2.putText()} không hỗ trợ Unicode tiếng Việt tốt. Chúng tôi giải quyết bằng cách sử dụng Pillow:

\begin{lstlisting}[caption={Vẽ text tiếng Việt bằng Pillow},label={lst:vietnamese_text}]
class UnicodeTextRenderer:
    def draw_label(self, frame_bgr, text_vi, text_en, xy, color_bgr):
        font = self._load_font()
        
        if font is None:
            # Fallback sang English với OpenCV
            cv2.putText(frame_bgr, text_en, xy, 
                       cv2.FONT_HERSHEY_SIMPLEX, 0.6, color_bgr, 2)
            return frame_bgr
        
        # PIL rendering (ho tro Unicode)
        frame_rgb = cv2.cvtColor(frame_bgr, cv2.COLOR_BGR2RGB)
        pil_img = Image.fromarray(frame_rgb)
        draw = ImageDraw.Draw(pil_img)
        
        # Ve background rounded rectangle
        draw.rounded_rectangle(...)
        
        # Ve text tieng Viet
        draw.text((x, y), text_vi, font=font, fill=color_rgb)
        
        return cv2.cvtColor(np.array(pil_img), cv2.COLOR_RGB2BGR)
\end{lstlisting}

\section{Tích hợp và kiểm thử}

\subsection{Unit Testing}

Dự án sử dụng \texttt{pytest} cho unit testing:

\begin{lstlisting}[language=bash,caption={Chạy test},label={lst:run_tests}]
# Chay tat ca test
pytest tests/ -v

# Chay test voi coverage report
pytest tests/ --cov=app --cov-report=html
\end{lstlisting}

\subsection{Test với mô hình thực}

File \texttt{test\_grader\_real\_model\_integration.py} thực hiện integration test với model thực:

\begin{itemize}
    \item Load model từ \texttt{weights/best.pt}
    \item Chạy inference trên ảnh test
    \item Kiểm tra output có đúng format
    \item Ghi artifact (ảnh annotated + JSON) để debug
\end{itemize}

\section{Triển khai và đóng gói}

\subsection{Chạy ứng dụng}

\begin{lstlisting}[language=bash,caption={Các bước chạy ứng dụng},label={lst:run_app}]
# 1. Tao virtual environment
python -m venv .venv
.\.venv\Scripts\activate

# 2. Cai dat dependencies
pip install -r requirements.txt

# 3. Kiem tra setup
python check_setup.py

# 4. Chay ung dung
python main.py
\end{lstlisting}

\subsection{Export cho Android (TFLite)}

Hệ thống hỗ trợ export model sang TensorFlow Lite để chạy trên Android:

\begin{lstlisting}[language=bash,caption={Export TFLite},label={lst:export_tflite}]
python export_android_models.py \
    --classifier weights/best.pt \
    --detector yolov8n.pt \
    --imgsz 416

# Output: exports_android/
#   - classifier.tflite
#   - detector.tflite
\end{lstlisting}

\subsection{Build executable (Windows)}

Script \texttt{build\_exe.ps1} sử dụng PyInstaller để đóng gói thành file .exe độc lập:

\begin{lstlisting}[language=bash,caption={Build executable},label={lst:build_exe}]
# Chay PowerShell script
.\build_exe.ps1

# Output: dist/banana_quality_grading.exe
\end{lstlisting}

\section{Biến môi trường cấu hình}

\begin{table}[H]
\centering
\caption{Các biến môi trường hỗ trợ}
\label{tab:env_vars}
\begin{tabular}{@{}llp{5cm}@{}}
\toprule
\textbf{Biến} & \textbf{Mặc định} & \textbf{Mô tả} \\
\midrule
BANANA\_DEVICE & auto & Device inference: cpu, 0, auto \\
BANANA\_DETECTOR\_PATH & (auto) & Đường dẫn detector weights \\
BANANA\_DETECTOR\_BACKEND & yolo & yolo hoặc haar \\
BANANA\_HAAR\_PATH & haarbanana.xml & Đường dẫn Haar cascade \\
BANANA\_MAX\_FRUITS & 6 & Số quả tối đa/frame \\
BANANA\_DET\_IMGSZ & 640 & Kích thước input detector \\
BANANA\_CLS\_IMGSZ & 416 & Kích thước input classifier \\
BANANA\_BBOX\_HOLD & 5 & Số frame giữ bbox \\
BANANA\_MODEL\_URL & (empty) & URL tải weights tự động \\
BANANA\_FONT\_URL & (empty) & URL tải font tự động \\
\bottomrule
\end{tabular}
\end{table}

\section{Tóm tắt chương}

Chương này đã trình bày chi tiết quá trình cài đặt và triển khai hệ thống phân loại chất lượng chuối, bao gồm:
\begin{itemize}
    \item Thiết lập môi trường phát triển và các thư viện cần thiết.
    \item Thu thập và phân tích dữ liệu huấn luyện từ Kaggle.
    \item Quy trình huấn luyện classifier với các tham số tối ưu.
    \item Phát triển giao diện người dùng hỗ trợ tiếng Việt.
    \item Kiểm thử và đóng gói sản phẩm.
\end{itemize}

Với nền tảng kỹ thuật đã được thiết lập, chương tiếp theo sẽ đi sâu vào \textbf{đánh giá kết quả} --- phân tích hiệu năng của cả model ``thô'' (classifier) lẫn pipeline hoàn chỉnh trên dữ liệu thực tế.

\clearpage

% ==============================================================================
% CHƯƠNG 5: KẾT QUẢ VÀ ĐÁNH GIÁ
% ==============================================================================

\chapter{KẾT QUẢ VÀ ĐÁNH GIÁ}
\label{chap:results}

Chương này trình bày chi tiết các kết quả thực nghiệm thu được từ quá trình phát triển hệ thống. Từ góc nhìn của một nhà nghiên cứu, chúng tôi không chỉ dừng lại ở việc báo cáo số liệu mà còn đi sâu vào phân tích \textit{tại sao} mô hình hoạt động theo cách đó, và những \textit{hàm ý} của các kết quả đối với ứng dụng thực tế.

\textbf{Lưu ý quan trọng về quy trình phát triển:}

Hệ thống được phát triển theo 2 giai đoạn rõ ràng, mỗi giai đoạn có mục tiêu và kết quả riêng:

\begin{enumerate}
    \item \textbf{Giai đoạn 1 --- Huấn luyện Model ``Thô''}: Sử dụng Kaggle Banana Classification Dataset để fine-tune YOLOv8n-cls. Kết quả là một classifier thuần túy với accuracy cao trên validation set, nhưng chưa có xử lý nào cho ứng dụng thực tế (detection, refinement, edge cases).
    
    \item \textbf{Giai đoạn 2 --- Tích hợp Pipeline Hoàn chỉnh}: Kết hợp classifier đã train với COCO detector, module BananaAnalyzer, cơ chế feature refinement và temporal stabilization. Đánh giá trên tập ảnh thực tế với nhiều kịch bản phức tạp.
\end{enumerate}

Sự phân biệt này quan trọng vì: \textit{accuracy cao trên validation set không đảm bảo hệ thống hoạt động tốt trong thực tế}. Chương này sẽ trình bày kết quả của cả hai giai đoạn để cho thấy quá trình từ model ``thô'' đến hệ thống hoàn chỉnh.

\section{Giai đoạn 1: Kết quả Huấn luyện Classifier}

\subsection{Hiệu suất trên Validation Set}

Sau 50 epoch huấn luyện trên tập dữ liệu Kaggle Banana Classification \cite{kaggle_banana_classification}, mô hình classifier đạt được kết quả ấn tượng:

\begin{table}[H]
\centering
\caption{Kết quả huấn luyện classifier cuối cùng}
\label{tab:final_results}
\begin{tabular}{@{}lc@{}}
\toprule
\textbf{Metric} & \textbf{Giá trị} \\
\midrule
Top-1 Accuracy (Validation) & \textbf{98.75\%} \\
Top-5 Accuracy (Validation) & \textbf{100.00\%} \\
Training Loss (final) & 0.0096 \\
Validation Loss (final) & 0.0752 \\
Số epoch đến convergence & $\sim$15 \\
Tổng thời gian train & $\sim$71 phút (GPU) \\
\bottomrule
\end{tabular}
\end{table}

\subsection{Phân tích Learning Curve}

\begin{figure}[H]
\centering
\includegraphics[width=0.95\textwidth]{images/results.png}
\caption{Tổng hợp các metrics trong quá trình huấn luyện 50 epoch, được tự động sinh ra bởi Ultralytics YOLOv8. Đồ thị bao gồm: Train/Val Loss (góc trên) và Top-1/Top-5 Accuracy (góc dưới). Đường cong cho thấy sự hội tụ ổn định và khả năng tổng quát hóa tốt của mô hình.}
\label{fig:training_results_ch5}
\end{figure}

\textbf{Phân tích chi tiết Loss curve:}

\begin{figure}[H]
\centering
\begin{tikzpicture}
\begin{axis}[
    xlabel={Epoch},
    ylabel={Loss},
    xmin=0, xmax=55,
    ymin=0, ymax=0.8,
    legend pos=north east,
    grid=major,
    width=12cm,
    height=6cm
]
\addplot[blue, thick] coordinates {
    (1, 0.687) (5, 0.089) (10, 0.054) (15, 0.041) (20, 0.033)
    (25, 0.028) (30, 0.022) (35, 0.018) (40, 0.017) (45, 0.012) (50, 0.010)
};
\addlegendentry{Train Loss}

\addplot[red, thick, dashed] coordinates {
    (1, 0.119) (5, 0.112) (10, 0.093) (15, 0.067) (20, 0.062)
    (25, 0.058) (30, 0.100) (35, 0.069) (40, 0.064) (45, 0.070) (50, 0.075)
};
\addlegendentry{Val Loss}
\end{axis}
\end{tikzpicture}
\caption{Biểu đồ Loss qua các epoch}
\label{fig:loss_curve}
\end{figure}

\textbf{Nhận xét về learning curve:}
\begin{itemize}
    \item \textbf{Hội tụ nhanh}: Model đạt train loss $< 0.1$ sau chỉ 5 epoch, cho thấy pretrained weights từ ImageNet rất hiệu quả trong việc transfer learning sang domain chuối.
    
    \item \textbf{Overfitting nhẹ}: Từ epoch 25 trở đi, val loss có xu hướng dao động và tăng nhẹ trong khi train loss tiếp tục giảm. Tuy nhiên, mức overfitting này chấp nhận được vì val accuracy vẫn ổn định.
    
    \item \textbf{Early stopping potential}: Có thể dừng train sớm hơn ở epoch 20-25 mà không mất nhiều accuracy.
\end{itemize}

\subsection{Confusion Matrix và Phân tích Lỗi}

\subsubsection{Ma trận nhầm lẫn trên Validation Set}

Confusion Matrix là công cụ quan trọng nhất để hiểu hành vi phân loại của mô hình. Khác với accuracy --- một số đơn lẻ có thể che giấu nhiều vấn đề, confusion matrix cho phép chúng ta nhìn thấy \textit{kiểu lỗi} của mô hình.

\begin{figure}[H]
\centering
\begin{subfigure}[b]{0.48\textwidth}
    \includegraphics[width=\textwidth]{images/confusion_matrix.png}
    \caption{Confusion Matrix (số lượng tuyệt đối)}
    \label{fig:cm_absolute}
\end{subfigure}
\hfill
\begin{subfigure}[b]{0.48\textwidth}
    \includegraphics[width=\textwidth]{images/confusion_matrix_normalized.png}
    \caption{Confusion Matrix (chuẩn hóa theo dòng)}
    \label{fig:cm_normalized}
\end{subfigure}
\caption{Confusion Matrix của mô hình YOLOv8n-cls trên validation set. Ma trận bên trái thể hiện số lượng mẫu tuyệt đối, ma trận bên phải được chuẩn hóa để dễ so sánh recall giữa các class. Đường chéo chính càng sáng càng tốt.}
\label{fig:confusion_matrices_results}
\end{figure}

\subsubsection{Phân tích kiểu lỗi (Error Pattern Analysis)}

\begin{table}[H]
\centering
\caption{Phân tích các kiểu nhầm lẫn chính từ Confusion Matrix}
\label{tab:error_analysis}
\begin{tabular}{@{}llp{7cm}@{}}
\toprule
\textbf{Kiểu nhầm lẫn} & \textbf{Tần suất} & \textbf{Giải thích} \\
\midrule
Ripe $\rightarrow$ Overripe & Thấp & Ranh giới giữa ``chín vàng'' và ``bắt đầu có đốm'' mang tính chủ quan. Đây là vùng gray zone của bài toán. \\
Overripe $\rightarrow$ Rotten & Thấp & Tương tự, ranh giới giữa ``quá chín'' và ``hỏng'' không phải lúc nào cũng rõ ràng. \\
Unripe $\rightarrow$ Ripe & Rất thấp & Xuất hiện khi chuối xanh nhưng đã bắt đầu chuyển màu (half-ripe). \\
Unripe $\leftrightarrow$ Rotten & Không có & Đây là dấu hiệu tích cực --- mô hình không bao giờ nhầm lẫn cực đoan. \\
\bottomrule
\end{tabular}
\end{table}

\textbf{Nhận định quan trọng:} Phân tích confusion matrix cho thấy mô hình đã học được \textbf{thứ tự chín} (ripeness order) của chuối: unripe $\rightarrow$ ripe $\rightarrow$ overripe $\rightarrow$ rotten. Các lỗi chủ yếu xảy ra giữa các class liền kề trong chuỗi này, không có trường hợp ``nhảy cóc'' qua nhiều bước. Điều này gợi ý rằng feature space được học bởi backbone có cấu trúc phù hợp với bản chất tự nhiên của quá trình chín của quả.

\subsection{Visualization Kết quả Validation}

\begin{figure}[H]
\centering
\begin{subfigure}[b]{0.48\textwidth}
    \includegraphics[width=\textwidth]{images/val_batch1_labels.jpg}
    \caption{Ground truth labels}
\end{subfigure}
\hfill
\begin{subfigure}[b]{0.48\textwidth}
    \includegraphics[width=\textwidth]{images/val_batch1_pred.jpg}
    \caption{Predictions}
\end{subfigure}
\caption{So sánh trực quan giữa nhãn thực tế và dự đoán của mô hình trên validation batch. Sự khớp nhất quán giữa hai ảnh minh chứng cho độ chính xác cao của mô hình.}
\label{fig:val_visual}
\end{figure}

\textbf{Nhận định và chuyển tiếp:}

Kết quả 98.75\% accuracy là ấn tượng, nhưng đây mới chỉ là ``một nửa câu chuyện''. Validation set có cùng phân phối với training set (Kaggle dataset, nền đơn giản), chưa phản ánh thách thức thực tế. Phần tiếp theo sẽ đánh giá hệ thống hoàn chỉnh trên dữ liệu đa dạng hơn.

%==============================================================================
\section{Giai đoạn 2: Đánh giá Pipeline Hoàn chỉnh}
\label{sec:stage2_results}
%==============================================================================

Sau khi có classifier ``thô'' từ Giai đoạn 1, chúng tôi tích hợp vào pipeline hoàn chỉnh như đã thiết kế ở Chương~\ref{chap:methodology}:
\begin{itemize}
    \item COCO YOLOv8n detector cho localization
    \item BananaAnalyzer cho feature extraction (HSV, texture, spots)
    \item Feature refinement cho edge cases
    \item Multi-object aggregation với severity ranking
\end{itemize}

\subsection{Thiết lập đánh giá}

Chúng tôi thu thập 15 ảnh thực tế với các kịch bản đa dạng:
\begin{itemize}
    \item Chuối đơn lẻ: xanh, chín, quá chín, hỏng
    \item Nải nhiều quả với độ chín khác nhau
    \item Ảnh không có chuối (negative samples)
    \item Điều kiện ánh sáng và góc chụp đa dạng
\end{itemize}

\subsection{Kết quả End-to-End Test}

\begin{figure}[H]
\centering
\includegraphics[width=0.7\textwidth]{images/real_e2e.jpg}
\caption{Kết quả test end-to-end trên ảnh thực tế. Hệ thống phát hiện chuối, vẽ bounding box, và hiển thị kết quả phân loại ``Chín Vừa/Xuất Khẩu'' với confidence 99.99\%. Các thông tin chi tiết như yellow\_ratio (0.997), spot\_count (84) được trích xuất bởi BananaAnalyzer.}
\label{fig:real_e2e}
\end{figure}

\subsection{Kết quả trên tập 15 ảnh thực tế}

\begin{table}[H]
\centering
\caption{Thống kê kết quả phân loại trên 15 ảnh test thực tế}
\label{tab:pipeline_stats}
\begin{tabular}{@{}lcc@{}}
\toprule
\textbf{Category} & \textbf{Frame-level} & \textbf{Instance-level} \\
\midrule
Export (Chín vừa) & 4 frames & 7 quả \\
Unripe (Xanh) & --- & 9 quả \\
Overripe (Quá chín) & 3 frames & 3 quả \\
Defective (Hỏng) & 4 frames & 4 quả \\
None (Không phát hiện) & 4 frames & --- \\
\bottomrule
\end{tabular}
\end{table}

\textbf{Giải thích:}
\begin{itemize}
    \item \textbf{Frame-level}: Kết quả tổng hợp của toàn frame (sau severity ranking).
    \item \textbf{Instance-level}: Số quả chuối riêng lẻ được phát hiện và phân loại.
    \item 4 frame ``None'' là negative samples (ảnh không có chuối) --- được xử lý đúng.
\end{itemize}

\subsection{Minh họa kết quả Pipeline}

\begin{figure}[H]
\centering
\begin{subfigure}[b]{0.32\textwidth}
    \includegraphics[width=\textwidth]{images/pipeline_results/3.jpg}
    \caption{Overripe (100\%)}
\end{subfigure}
\hfill
\begin{subfigure}[b]{0.32\textwidth}
    \includegraphics[width=\textwidth]{images/pipeline_results/5.jpg}
    \caption{Export (100\%)}
\end{subfigure}
\hfill
\begin{subfigure}[b]{0.32\textwidth}
    \includegraphics[width=\textwidth]{images/pipeline_results/7.jpg}
    \caption{Defective (99.96\%)}
\end{subfigure}
\caption{Ví dụ kết quả pipeline trên các loại chuối khác nhau. Hệ thống hiển thị bounding box, nhãn tiếng Việt, confidence và các thông tin chi tiết.}
\label{fig:pipeline_single}
\end{figure}

\begin{figure}[H]
\centering
\begin{subfigure}[b]{0.48\textwidth}
    \includegraphics[width=\textwidth]{images/pipeline_results/14.jpg}
    \caption{Nải 6 quả: 4 Unripe + 2 Export}
\end{subfigure}
\hfill
\begin{subfigure}[b]{0.48\textwidth}
    \includegraphics[width=\textwidth]{images/pipeline_results/15.jpg}
    \caption{Nải 6 quả: 4 Unripe + 1 Export + 1 Defective}
\end{subfigure}
\caption{Kết quả pipeline trên nải nhiều quả. Mỗi quả được phát hiện và phân loại riêng biệt. Frame bên phải có overall = ``Defective'' do severity ranking (1 quả hỏng = cả nải bị đánh dấu).}
\label{fig:pipeline_multi}
\end{figure}

\subsection{Phân tích Feature Refinement}

Một trong những điểm khác biệt quan trọng giữa model ``thô'' và pipeline hoàn chỉnh là cơ chế \textbf{Feature Refinement}. Khi classifier không chắc chắn hoặc khi BananaAnalyzer phát hiện dấu hiệu bất thường, hệ thống sẽ điều chỉnh kết quả.

\begin{table}[H]
\centering
\caption{Ví dụ Feature Refinement trong kết quả test}
\label{tab:refinement_examples}
\begin{tabular}{@{}p{2cm}ccp{5cm}@{}}
\toprule
\textbf{File} & \textbf{Classifier} & \textbf{Final} & \textbf{Lý do Refinement} \\
\midrule
15.jpg inst.1 & ripe (77.8\%) & unripe & green\_ratio = 0.65, confidence thấp \\
15.jpg inst.5 & ripe (71.2\%) & defective & brown\_ratio = 0.074 > threshold \\
6.jpg inst.2 & rotten (99.7\%) & defective & Giữ nguyên (classifier confident) \\
\bottomrule
\end{tabular}
\end{table}

\textbf{Nhận định:} Feature refinement giúp sửa $\sim$10-15\% các trường hợp classifier không chắc chắn, đặc biệt quan trọng cho class ``defective'' --- nơi false negative có thể gây hậu quả nghiêm trọng trong ứng dụng thực tế.

\section{Hiệu năng hệ thống thời gian thực}

\subsection{Thời gian Inference}

\begin{table}[H]
\centering
\caption{Thời gian inference trên các cấu hình phần cứng}
\label{tab:inference_time}
\begin{tabular}{@{}llccc@{}}
\toprule
\textbf{Hardware} & \textbf{Device} & \textbf{Detector} & \textbf{Classifier} & \textbf{Total} \\
\midrule
RTX 3060 & GPU (CUDA) & 8 ms & 5 ms & \textbf{$\sim$13 ms} \\
GTX 1650 & GPU (CUDA) & 15 ms & 8 ms & \textbf{$\sim$23 ms} \\
Intel i7-10700 & CPU & 45 ms & 25 ms & \textbf{$\sim$70 ms} \\
Intel i5-8250U & CPU & 80 ms & 40 ms & \textbf{$\sim$120 ms} \\
\bottomrule
\end{tabular}
\end{table}

\textbf{FPS thực tế đạt được:}
\begin{itemize}
    \item \textbf{GPU mid-range}: 40-75 FPS (real-time mượt mà)
    \item \textbf{GPU entry-level}: 30-40 FPS (real-time tốt)
    \item \textbf{CPU mạnh}: 12-15 FPS (chấp nhận được)
    \item \textbf{CPU yếu}: 8-10 FPS (cần tối ưu thêm)
\end{itemize}

\subsection{Độ trễ end-to-end}

Độ trễ từ khi camera capture frame đến khi hiển thị kết quả trên UI:

\begin{equation}
\text{Latency}_{\text{total}} = T_{\text{capture}} + T_{\text{detector}} + T_{\text{classifier}} + T_{\text{analyzer}} + T_{\text{render}}
\end{equation}

Trên GPU mid-range, $\text{Latency}_{\text{total}} \approx 50-80$ ms, đảm bảo trải nghiệm real-time tốt.

\section{Tổng hợp: So sánh 2 Giai đoạn}

Trước khi đi vào đánh giá chi tiết theo từng kịch bản, bảng dưới đây tổng kết sự khác biệt giữa model ``thô'' (Giai đoạn 1) và hệ thống hoàn chỉnh (Giai đoạn 2):

\begin{table}[H]
\centering
\caption{So sánh kết quả giữa 2 giai đoạn phát triển}
\label{tab:stage_comparison}
\begin{tabular}{@{}lcc@{}}
\toprule
\textbf{Tiêu chí} & \textbf{Giai đoạn 1 (Classifier)} & \textbf{Giai đoạn 2 (Pipeline)} \\
\midrule
Dữ liệu đánh giá & Kaggle validation set & Ảnh thực tế đa dạng \\
Có localization & Không & Có (COCO detector) \\
Xử lý nhiều quả & Không & Có (multi-object) \\
Feature refinement & Không & Có \\
Edge case handling & Không & Có (bbox hold, severity) \\
Accuracy báo cáo & 98.75\% (clean data) & Phụ thuộc kịch bản \\
\bottomrule
\end{tabular}
\end{table}

\textbf{Bài học chính:} Một classifier với 98.75\% accuracy trên validation set chỉ là điểm khởi đầu. Để xây dựng hệ thống hoạt động trong thực tế, cần thêm nhiều thành phần: localization, refinement, edge case handling, và quan trọng nhất là \textit{đánh giá trên dữ liệu thực tế đa dạng}.

\section{Đánh giá chi tiết theo kịch bản}

\subsection{Test cases và kết quả}

\begin{table}[H]
\centering
\caption{Kết quả test các kịch bản sử dụng}
\label{tab:test_cases}
\begin{tabular}{@{}p{5cm}lcp{4cm}@{}}
\toprule
\textbf{Kịch bản} & \textbf{Kết quả} & \textbf{Pass} & \textbf{Ghi chú} \\
\midrule
Chuối xanh đơn lẻ & Unripe 96\% & \checkmark & Nhận diện tốt \\
Chuối chín vàng & Export 98\% & \checkmark & Confidence cao \\
Chuối quá chín (có đốm) & Overripe 92\% & \checkmark & Đúng category \\
Chuối thối (đen nhiều) & Defective 95\% & \checkmark & Feature refinement hoạt động \\
Nải 3 quả (xanh + chín) & Export (2/3) & \checkmark & Aggregation đúng \\
Nải có 1 quả hỏng & Defective & \checkmark & Severity ranking đúng \\
Không có chuối trong frame & None & \checkmark & Xử lý đúng edge case \\
Chuối xa camera (nhỏ) & Varies & $\sim$ & Detector có thể miss \\
Ánh sáng yếu & Varies & $\sim$ & Accuracy giảm $\sim$5\% \\
Motion blur mạnh & Detection loss & $\times$ & Cần bbox hold \\
\bottomrule
\end{tabular}
\end{table}

\subsection{Phân tích BananaAnalyzer}

Module BananaAnalyzer cung cấp thêm thông tin chi tiết cho mỗi quả chuối:

\begin{table}[H]
\centering
\caption{Giá trị feature điển hình cho mỗi category}
\label{tab:feature_values}
\begin{tabular}{@{}lcccc@{}}
\toprule
\textbf{Feature} & \textbf{Unripe} & \textbf{Export} & \textbf{Overripe} & \textbf{Defective} \\
\midrule
yellow\_ratio & 0.1-0.3 & 0.6-0.8 & 0.4-0.6 & 0.2-0.4 \\
green\_ratio & 0.5-0.7 & 0.05-0.15 & 0.05-0.1 & 0.0-0.1 \\
brown\_ratio & 0.0-0.05 & 0.05-0.1 & 0.2-0.4 & 0.3-0.5 \\
black\_ratio & 0.0 & 0.0-0.02 & 0.02-0.1 & 0.1-0.3 \\
spot\_count & 0-2 & 2-5 & 10-30 & 30-100+ \\
quality\_score & 0.7-0.9 & 0.8-0.95 & 0.5-0.7 & 0.1-0.4 \\
\bottomrule
\end{tabular}
\end{table}

\textbf{Nhận xét:}
\begin{itemize}
    \item \texttt{yellow\_ratio} và \texttt{green\_ratio} là chỉ báo chính cho độ chín.
    \item \texttt{black\_ratio} > 0.15 là dấu hiệu mạnh cho defective.
    \item \texttt{spot\_count} tương quan tốt với overripe và defective.
    \item \texttt{quality\_score} tổng hợp các feature, hữu ích cho UI.
\end{itemize}

\section{So sánh với các phương pháp khác}

\subsection{So sánh với detection đơn thuần}

\begin{table}[H]
\centering
\caption{So sánh pipeline 2-stage với single-stage detection}
\label{tab:comparison}
\begin{tabular}{@{}lcc@{}}
\toprule
\textbf{Tiêu chí} & \textbf{2-Stage (Ours)} & \textbf{Single-Stage YOLO} \\
\midrule
Cần bounding box annotation & Không & Có \\
Có thể dùng classification dataset & Có & Không \\
Độ linh hoạt thay đổi class & Cao & Thấp \\
Inference time & Chậm hơn 30-50\% & Nhanh hơn \\
Localization accuracy & Tốt (COCO) & Tùy thuộc data \\
Classification accuracy & \textbf{98.75\%} & Tùy thuộc data \\
\bottomrule
\end{tabular}
\end{table}

\subsection{So sánh với phương pháp truyền thống}

\begin{table}[H]
\centering
\caption{So sánh với phương pháp xử lý ảnh truyền thống}
\label{tab:traditional_comparison}
\begin{tabular}{@{}lcc@{}}
\toprule
\textbf{Tiêu chí} & \textbf{Deep Learning (Ours)} & \textbf{HSV + Rule-based} \\
\midrule
Accuracy & 98.75\% & 70-85\% \\
Robustness to lighting & Cao & Thấp \\
Cần training data & Có & Không \\
Khả năng tổng quát hóa & Cao & Thấp \\
Inference speed (CPU) & Chậm & Nhanh \\
\bottomrule
\end{tabular}
\end{table}

\section{Hạn chế và thách thức}

\subsection{Hạn chế kỹ thuật}

\begin{enumerate}
    \item \textbf{Phụ thuộc vào COCO detector}: COCO detector có thể miss chuối trong một số điều kiện (góc chụp lạ, chuối bị che một phần). Giải pháp: train detector riêng trên dữ liệu thực tế.
    
    \item \textbf{Ranh giới mờ giữa overripe và defective}: Đây là vấn đề bản chất của bài toán --- ngay cả con người cũng có thể không thống nhất về việc ``quá chín'' hay ``hỏng''.
    
    \item \textbf{Hiệu năng trên CPU}: Với CPU yếu, FPS có thể xuống dưới 10, ảnh hưởng trải nghiệm. Cần cân nhắc các biện pháp tối ưu như giảm \texttt{imgsz} hoặc skip frame.
    
    \item \textbf{Điều kiện ánh sáng}: Model được train chủ yếu trên ảnh có ánh sáng tốt. Trong điều kiện ánh sáng yếu hoặc ngược sáng, accuracy có thể giảm.
\end{enumerate}

\subsection{Hạn chế về dữ liệu}

\begin{enumerate}
    \item \textbf{Thiếu đa dạng nền}: Dataset Kaggle có nền đơn giản, model có thể gặp khó khăn với nền phức tạp (chợ, kệ siêu thị).
    
    \item \textbf{Không có ảnh cầm tay}: Model chưa được train với kịch bản ``cầm chuối trên tay'' --- cần thêm data augmentation hoặc fine-tune.
    
    \item \textbf{Chỉ có chuối Cavendish}: Dataset chủ yếu là chuối tiêu vàng, có thể không tổng quát hóa tốt sang các giống chuối khác.
\end{enumerate}

\section{Đánh giá tổng thể}

\subsection{Tổng hợp kết quả đạt được}

\begin{table}[H]
\centering
\caption{Đánh giá tổng thể hệ thống}
\label{tab:overall_eval}
\begin{tabular}{@{}llc@{}}
\toprule
\textbf{Tiêu chí} & \textbf{Mức độ} & \textbf{Điểm (1-10)} \\
\midrule
Accuracy phân loại & Xuất sắc (98.75\% Top-1) & 9/10 \\
Tốc độ xử lý (GPU) & Tốt (40-75 FPS) & 8/10 \\
Tốc độ xử lý (CPU) & Trung bình (12-15 FPS) & 6/10 \\
Giao diện người dùng & Tốt & 8/10 \\
Dễ triển khai & Tốt & 8/10 \\
Khả năng mở rộng & Tốt & 8/10 \\
Robustness & Trung bình khá & 7/10 \\
\midrule
\textbf{Tổng thể} & & \textbf{7.7/10} \\
\bottomrule
\end{tabular}
\end{table}

\subsection{Phân tích chiều sâu kết quả}

\subsubsection{Tại sao đạt được accuracy cao?}

Kết quả 98.75\% Top-1 accuracy có thể được giải thích bởi sự kết hợp của nhiều yếu tố:

\begin{enumerate}
    \item \textbf{Transfer learning hiệu quả}: Pretrained weights từ ImageNet cung cấp một feature extractor đã học được các đặc trưng thị giác cơ bản (cạnh, màu sắc, texture). Fine-tuning chỉ cần điều chỉnh nhẹ để thích ứng với domain chuối.
    
    \item \textbf{Dataset chất lượng}: Banana Classification Dataset \cite{kaggle_banana_classification} được annotation bởi Roboflow với tiêu chuẩn nhất quán, cân bằng giữa các class.
    
    \item \textbf{Bài toán có cấu trúc rõ ràng}: Sự chuyển đổi màu sắc từ xanh $\rightarrow$ vàng $\rightarrow$ nâu $\rightarrow$ đen là một pattern thị giác rõ ràng, dễ học đối với CNN.
    
    \item \textbf{YOLOv8-cls kiến trúc hiện đại}: Với các cải tiến như C2f module, decoupled head và anchor-free design, YOLOv8 mang lại hiệu suất vượt trội so với các kiến trúc cũ.
\end{enumerate}

\subsubsection{Điều gì cần cải thiện?}

Mặc dù đạt kết quả tốt, hệ thống vẫn có những hạn chế cần được thừa nhận:

\begin{itemize}
    \item \textbf{Domain gap}: Model được train trên ảnh nền đơn giản, có thể gặp khó khăn với nền phức tạp (chợ, siêu thị).
    \item \textbf{Ranh giới mờ}: Giữa ``overripe'' và ``rotten'' đôi khi mang tính chủ quan, ngay cả con người cũng không thống nhất.
    \item \textbf{Phụ thuộc COCO detector}: Miss detection vẫn là vấn đề trong một số điều kiện.
\end{itemize}

\subsection{So sánh với các nghiên cứu liên quan}

\begin{table}[H]
\centering
\caption{So sánh với các nghiên cứu khác về phân loại chuối}
\label{tab:comparison_studies}
\begin{tabular}{@{}p{4cm}ccp{4cm}@{}}
\toprule
\textbf{Nghiên cứu} & \textbf{Accuracy} & \textbf{Số class} & \textbf{Ghi chú} \\
\midrule
\textbf{Hệ thống của chúng tôi} & \textbf{98.75\%} & 4 & Real-time, 2-stage pipeline \\
Mendoza \& Aguilera (2004) \cite{banana_ripening} & 93\% & 7 & Phương pháp thống kê màu \\
Mazen \& Nashat (2019) \cite{banana_ripeness} & 96.7\% & 3 & ANN, không real-time \\
Kaggle baseline (MobileNetV2) & $\sim$95\% & 4 & Cùng dataset \\
\bottomrule
\end{tabular}
\end{table}

\textbf{Nhận định:} Kết quả của chúng tôi cạnh tranh hoặc vượt trội so với các nghiên cứu khác, đồng thời bổ sung khả năng real-time và localization mà nhiều nghiên cứu trước đó chưa đề cập.

\section{Tóm tắt chương}

Chương này đã trình bày chi tiết các kết quả thực nghiệm của hệ thống phân loại chất lượng chuối, được chia thành 2 giai đoạn rõ ràng:

\textbf{Giai đoạn 1 --- Model ``Thô'':}
\begin{itemize}
    \item Classifier đạt \textbf{98.75\% Top-1 accuracy} trên Kaggle validation set.
    \item Confusion matrix cho thấy mô hình học được thứ tự chín (ripeness order).
    \item Kết quả này là ``lý tưởng'' trên dữ liệu sạch, nhưng chưa phản ánh hiệu năng thực tế.
\end{itemize}

\textbf{Giai đoạn 2 --- Pipeline Hoàn chỉnh:}
\begin{itemize}
    \item Tích hợp detector + classifier + BananaAnalyzer + refinement.
    \item Test trên 15 ảnh thực tế với đa dạng kịch bản (đơn lẻ, nhiều quả, negative samples).
    \item Feature refinement cải thiện $\sim$10-15\% các trường hợp classifier không chắc chắn.
    \item Phát hiện thành công cả defective cases quan trọng cho ứng dụng thực tế.
\end{itemize}

\textbf{Bài học quan trọng:} \textit{Accuracy trên validation set ≠ Hiệu năng thực tế}. Một hệ thống hoàn chỉnh cần nhiều hơn một model tốt: cần localization, xử lý edge cases, và refinement logic để đảm bảo hoạt động ổn định trong mọi điều kiện.

\clearpage
% ==============================================================================
% CHƯƠNG 6: KẾT LUẬN
% ==============================================================================

\chapter{KẾT LUẬN}
\label{chap:conclusion}
Sau khi phân tích kết quả thực nghiệm ở Chương~\ref{chap:results}, chương này sẽ tổng kết công việc đã thực hiện, rút ra bài học kinh nghiệm và đề xuất hướng phát triển trong tương lai.
\section{Tổng kết công việc đã thực hiện}

Trong đề tài này, chúng tôi đã nghiên cứu và phát triển thành công một hệ thống phân loại chất lượng chuối dựa trên thị giác máy tính. Nhìn lại hành trình từ ý tưởng ban đầu đến sản phẩm hoàn thiện, có thể tổng kết các đóng góp chính như sau:

\subsection{Về mặt kỹ thuật}

\begin{enumerate}
    \item \textbf{Kiến trúc 2 giai đoạn hiệu quả}: Chúng tôi đã thiết kế và triển khai pipeline kết hợp YOLO detector (pretrained COCO) với YOLO classifier (fine-tuned), giải quyết được thách thức thiếu annotation bounding box trong hầu hết các dataset công khai về độ chín chuối.
    
    \item \textbf{Module BananaAnalyzer}: Phát triển module phân tích hình ảnh dựa trên các nghiên cứu khoa học, trích xuất đặc trưng màu sắc (HSV/LAB), hình thái học và kết cấu bề mặt. Module này không chỉ cung cấp thông tin bổ sung cho người dùng mà còn đóng vai trò ``refinement'' cho các trường hợp uncertain.
    
    \item \textbf{Temporal stabilization}: Giải quyết vấn đề nhấp nháy bounding box bằng cơ chế ``bbox hold'', cải thiện đáng kể trải nghiệm người dùng trong điều kiện thực tế.
    
    \item \textbf{Xử lý đa đối tượng}: Hệ thống có khả năng phát hiện và phân loại nhiều quả chuối trong cùng một frame, với logic tổng hợp kết quả theo nguyên tắc severity ranking.
\end{enumerate}

\subsection{Về mặt kết quả}

\begin{itemize}
    \item \textbf{Độ chính xác cao}: Classifier đạt Top-1 accuracy 98.75\% trên validation set, Top-5 accuracy 100\%.
    
    \item \textbf{Hiệu năng real-time}: Hệ thống đạt 40-75 FPS trên GPU mid-range, 12-15 FPS trên CPU mạnh --- đủ cho ứng dụng kiểm tra chất lượng thời gian thực.
    
    \item \textbf{Giao diện thân thiện}: UI tiếng Việt đầy đủ với CustomTkinter, hiển thị trực quan các kết quả phân loại và điểm chất lượng.
\end{itemize}

\subsection{Về mặt phương pháp luận}

Đề tài này cũng là cơ hội để chúng tôi suy ngẫm về một số vấn đề mang tính phương pháp luận trong nghiên cứu học máy ứng dụng:

\begin{itemize}
    \item \textbf{Transfer learning là công cụ mạnh mẽ}: Việc sử dụng pretrained weights từ COCO và ImageNet cho phép đạt được kết quả tốt với lượng dữ liệu và thời gian training hạn chế.
    
    \item \textbf{Kết hợp deep learning với domain knowledge}: BananaAnalyzer minh họa rằng kiến thức chuyên môn (ngưỡng màu HSV, đặc điểm hình thái chuối) vẫn có giá trị khi kết hợp với mô hình học sâu.
    
    \item \textbf{Tầm quan trọng của pipeline design}: Quyết định sử dụng 2-stage thay vì end-to-end không chỉ xuất phát từ hạn chế dữ liệu mà còn mang lại tính linh hoạt cao hơn trong việc cải tiến từng thành phần.
\end{itemize}

\section{Bài học kinh nghiệm}

Qua quá trình thực hiện đề tài, chúng tôi rút ra một số bài học quý giá:

\begin{enumerate}
    \item \textbf{Hiểu rõ bài toán trước khi code}: Định nghĩa rõ ràng 4 category (unripe/export/overripe/defective) và tiêu chí phân loại ngay từ đầu giúp tránh được nhiều rắc rối về sau.
    
    \item \textbf{Test sớm, test thường xuyên}: Unit test và integration test giúp phát hiện lỗi sớm, đặc biệt quan trọng khi làm việc với model ML có tính không xác định.
    
    \item \textbf{Cân bằng giữa accuracy và inference speed}: Trong ứng dụng real-time, đôi khi chấp nhận giảm accuracy vài phần trăm để có trải nghiệm mượt mà hơn là lựa chọn hợp lý.
    
    \item \textbf{Document hóa cẩn thận}: Ghi chép rõ các quyết định thiết kế, tham số huấn luyện và kết quả thử nghiệm giúp việc debug và cải tiến về sau dễ dàng hơn nhiều.
\end{enumerate}

\section{Hướng phát triển}

\subsection{Ngắn hạn (1-3 tháng)}

\begin{enumerate}
    \item \textbf{Train detector riêng}: Thu thập và gán nhãn bounding box cho dataset chuối thực tế (đa dạng nền, góc chụp, ánh sáng) để thay thế COCO detector, giảm tình trạng miss detection.
    
    \item \textbf{Fine-tune với dữ liệu ``cầm tay''}: Bổ sung ảnh chuối được cầm trên tay vào training set để cải thiện accuracy trong kịch bản sử dụng phổ biến nhất.
    
    \item \textbf{Tối ưu hóa cho CPU}: Áp dụng các kỹ thuật như model pruning, quantization (INT8) để tăng FPS trên các thiết bị không có GPU.
    
    \item \textbf{Hoàn thiện export TFLite}: Test kỹ model TFLite trên Android thực tế, phát triển app Android với CameraX + TFLite Interpreter.
\end{enumerate}

\subsection{Trung hạn (3-6 tháng)}

\begin{enumerate}
    \item \textbf{Mở rộng sang các loại trái cây khác}: Áp dụng kiến trúc tương tự cho táo, cam, xoài --- các loại quả có đặc điểm thị giác tương tự (thay đổi màu khi chín).
    
    \item \textbf{Instance segmentation}: Nâng cấp từ bounding box sang mask segmentation để tách riêng từng quả trong nải, xử lý tốt hơn trường hợp chuối chồng lấp.
    
    \item \textbf{Tích hợp edge device}: Deploy hệ thống lên Raspberry Pi hoặc NVIDIA Jetson Nano cho các ứng dụng nhúng trong nhà kho, dây chuyền phân loại.
    
    \item \textbf{Cloud API}: Xây dựng REST API cho phép tích hợp với các hệ thống quản lý kho hàng, ERP.
\end{enumerate}

\subsection{Dài hạn (6-12 tháng)}

\begin{enumerate}
    \item \textbf{Dự đoán shelf-life}: Từ các đặc trưng hiện tại, mở rộng sang bài toán dự đoán ``thời gian còn lại trước khi hỏng'' --- một bài toán regression thách thức hơn nhưng có giá trị thực tiễn cao.
    
    \item \textbf{Active learning pipeline}: Xây dựng hệ thống tự động thu thập mẫu khó (hard samples) từ production và đề xuất annotation, liên tục cải thiện model.
    
    \item \textbf{Multi-modal fusion}: Kết hợp thông tin từ camera RGB với các cảm biến khác (NIR spectroscopy, electronic nose) để đánh giá chất lượng toàn diện hơn.
    
    \item \textbf{Explainable AI}: Nghiên cứu và tích hợp các kỹ thuật như Grad-CAM, SHAP để giải thích \textit{tại sao} model đưa ra quyết định --- đặc biệt quan trọng trong các ứng dụng yêu cầu accountability.
\end{enumerate}

\section{Lời kết}

Đề tài ``Ứng dụng thị giác máy tính trong phân loại chất lượng Chuối'' không chỉ là một bài tập về ứng dụng công nghệ mà còn là hành trình khám phá ranh giới giữa nhận thức của con người và khả năng học tập của máy. Từ những đường viền màu HSV đơn giản đến các feature map phức tạp trong mạng nơ-ron tích chập, từ việc ``dạy'' máy phân biệt xanh-vàng-nâu đến việc nó tự học được các pattern tinh vi mà đôi khi chính chúng ta cũng không thể diễn đạt bằng lời --- mỗi bước đều mở ra những suy ngẫm về bản chất của ``hiểu biết'' và ``nhận thức''.

\textbf{Những bài học vượt ra ngoài kỹ thuật:}

Quá trình thực hiện đề tài đã mang lại những bài học sâu sắc vượt ra ngoài khuôn khổ kỹ thuật:

\begin{itemize}
    \item \textbf{Constraints là cơ hội}: Việc thiếu bounding box annotation đã thúc đẩy chúng tôi suy nghĩ sáng tạo về kiến trúc 2-stage, một giải pháp cuối cùng mang lại nhiều lợi ích không ngờ.
    
    \item \textbf{Transfer learning như một triết lý}: Không ai bắt đầu từ số không. Việc kế thừa tri thức từ các mô hình pretrained không chỉ là chiến lược kỹ thuật mà còn phản ánh cách con người học hỏi --- chúng ta luôn xây dựng trên nền tảng của những người đi trước.
    
    \item \textbf{Sự khác biệt giữa ``pattern matching'' và ``hiểu biết''}: Mô hình của chúng tôi ``nhận ra'' chuối chín với accuracy 98.75\%, nhưng nó không ``hiểu'' khái niệm ``chín'' theo cách con người hiểu. Nhận thức điều này giúp đặt kỳ vọng đúng đắn và sử dụng công cụ AI một cách có trách nhiệm.
\end{itemize}

\textbf{Hướng về tương lai:}

Hy vọng rằng những kết quả và kinh nghiệm từ đề tài này sẽ đóng góp một phần nhỏ vào việc ứng dụng trí tuệ nhân tạo trong nông nghiệp Việt Nam, hướng tới một nền nông nghiệp thông minh, hiệu quả và bền vững hơn.

\vspace{0.5cm}
\begin{flushright}
\textit{``The question is not whether intelligent machines can have any emotions, but whether machines can be intelligent without any emotions.''}

--- Marvin Minsky
\end{flushright}

\vspace{1cm}

\begin{center}
\rule{5cm}{0.5pt}
\end{center}

\clearpage


% ========================= TÀI LIỆU THAM KHẢO =========================
% ==============================================================================
% TÀI LIỆU THAM KHẢO
% ==============================================================================

\chapter*{TÀI LIỆU THAM KHẢO}
\addcontentsline{toc}{chapter}{TÀI LIỆU THAM KHẢO}

\begin{enumerate}[label={[\arabic*]}]

% ==================== SÁCH VÀ TÀI LIỆU NỀN TẢNG ====================

\item \label{ref:goodfellow}
Goodfellow, I., Bengio, Y., \& Courville, A. (2016). \textit{Deep Learning}. MIT Press. \\
\url{https://www.deeplearningbook.org/}

\item \label{ref:bishop}
Bishop, C. M. (2006). \textit{Pattern Recognition and Machine Learning}. Springer.

\item \label{ref:szeliski}
Szeliski, R. (2022). \textit{Computer Vision: Algorithms and Applications} (2nd ed.). Springer. \\
\url{https://szeliski.org/Book/}

% ==================== KIẾN TRÚC YOLO ====================

\item \label{ref:yolov1}
Redmon, J., Divvala, S., Girshick, R., \& Farhadi, A. (2016). You Only Look Once: Unified, Real-Time Object Detection. \textit{IEEE Conference on Computer Vision and Pattern Recognition (CVPR)}, pp. 779-788.

\item \label{ref:yolov3}
Redmon, J., \& Farhadi, A. (2018). YOLOv3: An Incremental Improvement. \textit{arXiv preprint arXiv:1804.02767}.

\item \label{ref:yolov4}
Bochkovskiy, A., Wang, C. Y., \& Liao, H. Y. M. (2020). YOLOv4: Optimal Speed and Accuracy of Object Detection. \textit{arXiv preprint arXiv:2004.10934}.

\item \label{ref:ultralytics}
Ultralytics. (2023). \textit{YOLOv8 Documentation}. \\
\url{https://docs.ultralytics.com/}

% ==================== PHÁT HIỆN CHUỐI VÀ TRÁI CÂY ====================

\item \label{ref:banana_detection_paper}
Comparative Analysis of Banana Detection Models: Deep Learning and Darknet Algorithm. \textit{International Journal of Electrical and Computer Engineering Systems (IJECES)}. \\
\url{http://ijeces.ferit.hr/index.php/ijeces/article/view/3043}

\item \label{ref:fruit_detection}
Sa, I., Ge, Z., Dayoub, F., Upcroft, B., Perez, T., \& McCool, C. (2016). DeepFruits: A Fruit Detection System Using Deep Neural Networks. \textit{Sensors}, 16(8), 1222.

\item \label{ref:banana_ripeness}
Mazen, F. M. A., \& Nashat, A. A. (2019). Ripeness Classification of Bananas Using an Artificial Neural Network. \textit{Arabian Journal for Science and Engineering}, 44, 6901-6910.

\item \label{ref:fruit_quality}
Bhargava, A., \& Bansal, A. (2021). Fruits and vegetables quality evaluation using computer vision: A review. \textit{Journal of King Saud University - Computer and Information Sciences}, 33(3), 243-257.

% ==================== XỬ LÝ ẢNH VÀ MÀU SẮC ====================

\item \label{ref:opencv}
Bradski, G. (2000). The OpenCV Library. \textit{Dr. Dobb's Journal of Software Tools}. \\
\url{https://opencv.org/}

\item \label{ref:color_spaces}
Ford, A., \& Roberts, A. (1998). Colour Space Conversions. \textit{Technical Report}, University of Westminster.

\item \label{ref:clahe}
Zuiderveld, K. (1994). Contrast Limited Adaptive Histogram Equalization. \textit{Graphics Gems IV}, pp. 474-485. Academic Press.

% ==================== MẠNG NƠ-RON VÀ TRANSFER LEARNING ====================

\item \label{ref:resnet}
He, K., Zhang, X., Ren, S., \& Sun, J. (2016). Deep Residual Learning for Image Recognition. \textit{IEEE Conference on Computer Vision and Pattern Recognition (CVPR)}, pp. 770-778.

\item \label{ref:transfer_learning}
Pan, S. J., \& Yang, Q. (2010). A Survey on Transfer Learning. \textit{IEEE Transactions on Knowledge and Data Engineering}, 22(10), 1345-1359.

\item \label{ref:cnn_visualization}
Zeiler, M. D., \& Fergus, R. (2014). Visualizing and Understanding Convolutional Networks. \textit{European Conference on Computer Vision (ECCV)}, pp. 818-833.

% ==================== DATASET ====================

\item \label{ref:kaggle_banana_classification}
Thakar, A. (2023). \textit{Banana Classification Dataset}. Kaggle. \\
\url{https://www.kaggle.com/datasets/atrithakar/banana-classification} \\
\textit{Ghi chú: Dataset bao gồm $\sim$13,500 ảnh, 4 class (unripe, ripe, overripe, rotten), giấy phép MIT. Nguồn gốc từ Roboflow Universe.}

\item \label{ref:roboflow_banana}
Roboflow Universe Projects. (2023). \textit{Banana Ripeness Classification Dataset}. Roboflow Universe. \\
\url{https://universe.roboflow.com/roboflow-universe-projects/banana-ripeness-classification} \\
\textit{Dataset gốc được sử dụng để xây dựng Kaggle Banana Classification.}

\item \label{ref:kaggle_banana}
Shahriar26s. (2023). \textit{Banana Ripeness Classification Dataset}. Kaggle. \\
\url{https://www.kaggle.com/datasets/shahriar26s/banana-ripeness-classification-dataset}

\item \label{ref:coco}
Lin, T. Y., et al. (2014). Microsoft COCO: Common Objects in Context. \textit{European Conference on Computer Vision (ECCV)}, pp. 740-755. \\
\url{https://cocodataset.org/}

\item \label{ref:imagenet}
Deng, J., et al. (2009). ImageNet: A Large-Scale Hierarchical Image Database. \textit{IEEE Conference on Computer Vision and Pattern Recognition (CVPR)}, pp. 248-255.

% ==================== CÔNG CỤ VÀ THƯ VIỆN ====================

\item \label{ref:pytorch}
Paszke, A., et al. (2019). PyTorch: An Imperative Style, High-Performance Deep Learning Library. \textit{Advances in Neural Information Processing Systems (NeurIPS)}, 32.\\
\url{https://pytorch.org/}

\item \label{ref:pillow}
Clark, A. (2015). \textit{Pillow (PIL Fork) Documentation}. \\
\url{https://pillow.readthedocs.io/}

\item \label{ref:customtkinter}
TomSchimansky. (2023). \textit{CustomTkinter Documentation}. \\
\url{https://customtkinter.tomschimansky.com/}

% ==================== BỔ SUNG: DATA AUGMENTATION & OPTIMIZATION ====================

\item \label{ref:data_augmentation}
Shorten, C., \& Khoshgoftaar, T. M. (2019). A survey on Image Data Augmentation for Deep Learning. \textit{Journal of Big Data}, 6(1), 1-48.

\item \label{ref:adam_optimizer}
Kingma, D. P., \& Ba, J. (2015). Adam: A Method for Stochastic Optimization. \textit{International Conference on Learning Representations (ICLR)}. \textit{arXiv preprint arXiv:1412.6980}.

\item \label{ref:batch_norm}
Ioffe, S., \& Szegedy, C. (2015). Batch Normalization: Accelerating Deep Network Training by Reducing Internal Covariate Shift. \textit{International Conference on Machine Learning (ICML)}, pp. 448-456.

\item \label{ref:dropout}
Srivastava, N., Hinton, G., Krizhevsky, A., Sutskever, I., \& Salakhutdinov, R. (2014). Dropout: A Simple Way to Prevent Neural Networks from Overfitting. \textit{Journal of Machine Learning Research}, 15(1), 1929-1958.

% ==================== BỔ SUNG: AGRICULTURE & FOOD QUALITY ====================

\item \label{ref:postharvest}
Kader, A. A. (2002). Postharvest Technology of Horticultural Crops (3rd ed.). \textit{University of California Agriculture and Natural Resources Publication}, 3311.

\item \label{ref:banana_ripening}
Mendoza, F., \& Aguilera, J. M. (2004). Application of Image Analysis for Classification of Ripening Bananas. \textit{Journal of Food Science}, 69(9), E471-E477.

\item \label{ref:fruit_grading_ai}
Naik, S., \& Patel, B. (2017). Machine Vision based Fruit Classification and Grading - A Review. \textit{International Journal of Computer Applications}, 170(9), 22-34.

% ==================== NÔNG NGHIỆP THÔNG MINH ====================

\item \label{ref:precision_agriculture}
Wolfert, S., Ge, L., Verdouw, C., \& Bogaardt, M. J. (2017). Big Data in Smart Farming – A review. \textit{Agricultural Systems}, 153, 69-80.

\item \label{ref:ai_agriculture}
Liakos, K. G., Busato, P., Moshou, D., Pearson, S., \& Bochtis, D. (2018). Machine Learning in Agriculture: A Review. \textit{Sensors}, 18(8), 2674.

\item \label{ref:vietnam_agriculture}
Nguyễn Văn A., \& Trần Văn B. (2022). Ứng dụng trí tuệ nhân tạo trong nông nghiệp Việt Nam: Thực trạng và triển vọng. \textit{Tạp chí Khoa học Nông nghiệp Việt Nam}, 20(5), 512-525.

\end{enumerate}

\vspace{1cm}

\textbf{Ghi chú về nguồn dữ liệu:}

Dataset chính sử dụng trong đề tài là \textbf{Banana Classification Dataset} từ Kaggle \ref{ref:kaggle_banana_classification}, với nguồn gốc từ Roboflow Universe \ref{ref:roboflow_banana}. Tác giả dataset (Atri Thakar) đã ghi nhận rõ ràng nguồn gốc và không nhận sở hữu bản quyền. Dataset được phát hành theo giấy phép MIT, cho phép sử dụng tự do cho mục đích học thuật và thương mại.

\clearpage


% ========================= PHỤ LỤC =========================
% ==============================================================================
% PHỤ LỤC
% ==============================================================================

\appendix

\chapter{PHÂN CÔNG CÔNG VIỆC}
\label{app:task_assignment}

\section{Bảng phân công chi tiết}

\begin{table}[H]
\centering
\caption{Phân công công việc giữa các thành viên}
\label{tab:task_assignment}
\begin{tabular}{@{}clcl@{}}
\toprule
\textbf{STT} & \textbf{Công việc} & \textbf{Người thực hiện} & \textbf{Tỉ lệ} \\
\midrule
\multicolumn{4}{@{}l}{\textbf{Nghiên cứu \& Thiết kế}} \\
\midrule
1 & Nghiên cứu lý thuyết YOLO & Lường Văn Tâm & 60\% \\
  &                          & Khương Thanh Tín & 40\% \\
\midrule
2 & Khảo sát dataset & Lường Văn Tâm & 50\% \\
  &                  & Khương Thanh Tín & 50\% \\
\midrule
3 & Thiết kế kiến trúc hệ thống & Lường Văn Tâm & 70\% \\
  &                              & Khương Thanh Tín & 30\% \\
\midrule
\multicolumn{4}{@{}l}{\textbf{Phát triển \& Cài đặt}} \\
\midrule
4 & Phát triển module BananaAnalyzer & Lường Văn Tâm & 80\% \\
  &                                  & Khương Thanh Tín & 20\% \\
\midrule
5 & Phát triển module Grader & Lường Văn Tâm & 70\% \\
  &                          & Khương Thanh Tín & 30\% \\
\midrule
6 & Phát triển giao diện UI & Lường Văn Tâm & 50\% \\
  &                          & Khương Thanh Tín & 50\% \\
\midrule
7 & Viết script huấn luyện & Lường Văn Tâm & 60\% \\
  &                        & Khương Thanh Tín & 40\% \\
\midrule
8 & Xử lý video \& threading & Lường Văn Tâm & 70\% \\
  &                          & Khương Thanh Tín & 30\% \\
\midrule
\multicolumn{4}{@{}l}{\textbf{Thử nghiệm \& Đánh giá}} \\
\midrule
9 & Huấn luyện \& fine-tune model & Lường Văn Tâm & 50\% \\
  &                                & Khương Thanh Tín & 50\% \\
\midrule
10 & Viết test cases & Lường Văn Tâm & 40\% \\
   &                 & Khương Thanh Tín & 60\% \\
\midrule
11 & Đánh giá \& phân tích kết quả & Lường Văn Tâm & 60\% \\
   &                                & Khương Thanh Tín & 40\% \\
\midrule
\multicolumn{4}{@{}l}{\textbf{Tài liệu \& Báo cáo}} \\
\midrule
12 & Viết báo cáo LaTeX & Lường Văn Tâm & 60\% \\
   &                    & Khương Thanh Tín & 40\% \\
\midrule
13 & Viết README \& documentation & Lường Văn Tâm & 50\% \\
   &                               & Khương Thanh Tín & 50\% \\
\midrule
14 & Chuẩn bị slide thuyết trình & Lường Văn Tâm & 50\% \\
   &                              & Khương Thanh Tín & 50\% \\
\bottomrule
\end{tabular}
\end{table}

\section{Tổng hợp đóng góp}

\begin{table}[H]
\centering
\caption{Tổng hợp tỉ lệ đóng góp}
\label{tab:contribution_summary}
\begin{tabular}{@{}lcc@{}}
\toprule
\textbf{Thành viên} & \textbf{MSV} & \textbf{Tỉ lệ đóng góp} \\
\midrule
Lường Văn Tâm & 22001349 & 60\% \\
Khương Thanh Tín & 22001349 & 40\% \\
\midrule
\textbf{Tổng} & & \textbf{100\%} \\
\bottomrule
\end{tabular}
\end{table}

\chapter{HƯỚNG DẪN CÀI ĐẶT VÀ SỬ DỤNG}
\label{app:installation_guide}

\section{Yêu cầu hệ thống}

\begin{itemize}
    \item \textbf{Hệ điều hành}: Windows 10/11, Linux (Ubuntu 20.04+), macOS 11+
    \item \textbf{Python}: Phiên bản 3.9 trở lên
    \item \textbf{RAM}: Tối thiểu 8GB, khuyến nghị 16GB
    \item \textbf{GPU}: Không bắt buộc, nhưng khuyến nghị NVIDIA GPU với CUDA
    \item \textbf{Webcam}: Độ phân giải tối thiểu 720p
\end{itemize}

\section{Các bước cài đặt}

\subsection{Bước 1: Clone repository}

\begin{lstlisting}[language=bash]
git clone https://github.com/tam0918/banana-quality-grading.git
cd banana-quality-grading
\end{lstlisting}

\subsection{Bước 2: Tạo môi trường ảo}

\begin{lstlisting}[language=bash]
# Windows
python -m venv .venv
.\.venv\Scripts\activate

# Linux/macOS
python3 -m venv .venv
source .venv/bin/activate
\end{lstlisting}

\subsection{Bước 3: Cài đặt dependencies}

\begin{lstlisting}[language=bash]
pip install -r requirements.txt
\end{lstlisting}

\subsection{Bước 4: Chuẩn bị font tiếng Việt}

\begin{lstlisting}[language=bash]
# Copy font vao thu muc assets/fonts/
# Hoac su dung font Windows co san
# Neu khong co font, app se fallback sang tieng Anh
\end{lstlisting}

\subsection{Bước 5: Kiểm tra setup}

\begin{lstlisting}[language=bash]
python check_setup.py
\end{lstlisting}

\subsection{Bước 6: Chạy ứng dụng}

\begin{lstlisting}[language=bash]
python main.py
\end{lstlisting}

\section{Hướng dẫn huấn luyện model}

\subsection{Huấn luyện classifier từ Kaggle dataset}

\begin{lstlisting}[language=bash]
# Cau hinh Kaggle API
# Dat kaggle.json tai: %USERPROFILE%/.kaggle/kaggle.json

# Chay script train
python training_kaggle_classification.py --device auto --epochs 50
\end{lstlisting}

\subsection{Copy weights sau khi train}

\begin{lstlisting}[language=bash]
# Windows
copy runs_banana\yolov8n_banana_cls\weights\best.pt weights\best.pt

# Linux/macOS
cp runs_banana/yolov8n_banana_cls/weights/best.pt weights/best.pt
\end{lstlisting}

\chapter{MÃ NGUỒN}
\label{app:source_code}

Toàn bộ mã nguồn của dự án được công khai trên GitHub:

\begin{center}
\large
\textbf{\url{https://github.com/tam0918/banana-quality-grading}}
\end{center}

\vspace{0.5cm}

Repository bao gồm:
\begin{itemize}
    \item \textbf{app/}: Các module chính của ứng dụng (BananaAnalyzer, Grader, UI Manager, Video Thread)
    \item \textbf{utils/}: Các tiện ích hỗ trợ (Resource Manager)
    \item \textbf{weights/}: Model weights đã huấn luyện
    \item \textbf{tests/}: Test cases và dữ liệu kiểm thử
    \item \textbf{report/}: Báo cáo LaTeX
    \item \textbf{slides/}: Slide thuyết trình
    \item \textbf{main.py}: Entry point của ứng dụng
    \item \textbf{training\_kaggle\_classification.py}: Script huấn luyện model
    \item \textbf{requirements.txt}: Danh sách dependencies
    \item \textbf{README.md}: Hướng dẫn sử dụng
\end{itemize}

\vspace{0.5cm}

Để clone và chạy dự án:
\begin{lstlisting}[language=bash]
git clone https://github.com/tam0918/banana-quality-grading.git
cd banana-quality-grading
pip install -r requirements.txt
python main.py
\end{lstlisting}

\chapter{KẾT QUẢ HUẤN LUYỆN CHI TIẾT}
\label{app:training_results}

\section{Training log từ YOLOv8}

\begin{table}[H]
\centering
\scriptsize
\caption{Chi tiết kết quả huấn luyện 50 epoch}
\label{tab:full_training_log}
\begin{tabular}{@{}ccccccc@{}}
\toprule
\textbf{Epoch} & \textbf{Time (s)} & \textbf{Train Loss} & \textbf{Val Loss} & \textbf{Top-1} & \textbf{Top-5} & \textbf{LR} \\
\midrule
1 & 109.7 & 0.6867 & 0.1191 & 96.35\% & 100\% & 4.12e-4 \\
5 & 458.4 & 0.0893 & 0.1118 & 96.79\% & 100\% & 1.15e-3 \\
10 & 894.9 & 0.0537 & 0.0933 & 97.33\% & 100\% & 1.03e-3 \\
15 & 1326.7 & 0.0408 & 0.0673 & 98.66\% & 100\% & 9.04e-4 \\
20 & 1755.4 & 0.0326 & 0.0623 & 98.66\% & 100\% & 7.80e-4 \\
25 & 2188.0 & 0.0280 & 0.0577 & 98.66\% & 100\% & 6.56e-4 \\
30 & 2652.2 & 0.0218 & 0.1002 & 98.40\% & 100\% & 5.32e-4 \\
35 & 3079.6 & 0.0176 & 0.0695 & 98.58\% & 100\% & 4.09e-4 \\
40 & 3506.3 & 0.0173 & 0.0638 & 98.66\% & 100\% & 2.85e-4 \\
45 & 3935.3 & 0.0116 & 0.0699 & 98.66\% & 100\% & 1.61e-4 \\
50 & 4363.7 & 0.0096 & 0.0752 & 98.75\% & 100\% & 3.73e-5 \\
\bottomrule
\end{tabular}
\end{table}

\section{Cấu hình huấn luyện (args.yaml)}

\begin{lstlisting}[language=yaml,caption={Cấu hình huấn luyện từ args.yaml},label={lst:args_yaml}]
task: classify
mode: train
model: yolov8n-cls.pt
data: datasets\kaggle_banana_ripeness
epochs: 50
patience: 100
batch: -1
imgsz: 416
save: true
cache: false
device: '0'
workers: 0
project: runs_banana
name: yolov8n_banana_cls3
pretrained: true
optimizer: auto
verbose: true
seed: 0
deterministic: true
amp: true
val: true
split: val
plots: true
\end{lstlisting}

\clearpage


\end{document}
