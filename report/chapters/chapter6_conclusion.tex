% ==============================================================================
% CHƯƠNG 6: KẾT LUẬN
% ==============================================================================

\chapter{KẾT LUẬN}
\label{chap:conclusion}
Sau khi phân tích kết quả thực nghiệm ở Chương~\ref{chap:results}, chương này sẽ tổng kết công việc đã thực hiện, rút ra bài học kinh nghiệm và đề xuất hướng phát triển trong tương lai.
\section{Tổng kết công việc đã thực hiện}

Trong đề tài này, chúng tôi đã nghiên cứu và phát triển thành công một hệ thống phân loại chất lượng chuối dựa trên thị giác máy tính. Nhìn lại hành trình từ ý tưởng ban đầu đến sản phẩm hoàn thiện, có thể tổng kết các đóng góp chính như sau:

\subsection{Về mặt kỹ thuật}

\begin{enumerate}
    \item \textbf{Kiến trúc 2 giai đoạn hiệu quả}: Chúng tôi đã thiết kế và triển khai pipeline kết hợp YOLO detector (pretrained COCO) với YOLO classifier (fine-tuned), giải quyết được thách thức thiếu annotation bounding box trong hầu hết các dataset công khai về độ chín chuối.
    
    \item \textbf{Module BananaAnalyzer}: Phát triển module phân tích hình ảnh dựa trên các nghiên cứu khoa học, trích xuất đặc trưng màu sắc (HSV/LAB), hình thái học và kết cấu bề mặt. Module này không chỉ cung cấp thông tin bổ sung cho người dùng mà còn đóng vai trò ``refinement'' cho các trường hợp uncertain.
    
    \item \textbf{Temporal stabilization}: Giải quyết vấn đề nhấp nháy bounding box bằng cơ chế ``bbox hold'', cải thiện đáng kể trải nghiệm người dùng trong điều kiện thực tế.
    
    \item \textbf{Xử lý đa đối tượng}: Hệ thống có khả năng phát hiện và phân loại nhiều quả chuối trong cùng một frame, với logic tổng hợp kết quả theo nguyên tắc severity ranking.
\end{enumerate}

\subsection{Về mặt kết quả}

\begin{itemize}
    \item \textbf{Độ chính xác cao}: Classifier đạt Top-1 accuracy 98.75\% trên validation set, Top-5 accuracy 100\%.
    
    \item \textbf{Hiệu năng real-time}: Hệ thống đạt 40-75 FPS trên GPU mid-range, 12-15 FPS trên CPU mạnh --- đủ cho ứng dụng kiểm tra chất lượng thời gian thực.
    
    \item \textbf{Giao diện thân thiện}: UI tiếng Việt đầy đủ với CustomTkinter, hiển thị trực quan các kết quả phân loại và điểm chất lượng.
\end{itemize}

\subsection{Về mặt phương pháp luận}

Đề tài này cũng là cơ hội để chúng tôi suy ngẫm về một số vấn đề mang tính phương pháp luận trong nghiên cứu học máy ứng dụng:

\begin{itemize}
    \item \textbf{Transfer learning là công cụ mạnh mẽ}: Việc sử dụng pretrained weights từ COCO và ImageNet cho phép đạt được kết quả tốt với lượng dữ liệu và thời gian training hạn chế.
    
    \item \textbf{Kết hợp deep learning với domain knowledge}: BananaAnalyzer minh họa rằng kiến thức chuyên môn (ngưỡng màu HSV, đặc điểm hình thái chuối) vẫn có giá trị khi kết hợp với mô hình học sâu.
    
    \item \textbf{Tầm quan trọng của pipeline design}: Quyết định sử dụng 2-stage thay vì end-to-end không chỉ xuất phát từ hạn chế dữ liệu mà còn mang lại tính linh hoạt cao hơn trong việc cải tiến từng thành phần.
\end{itemize}

\section{Bài học kinh nghiệm}

Qua quá trình thực hiện đề tài, chúng tôi rút ra một số bài học quý giá:

\begin{enumerate}
    \item \textbf{Hiểu rõ bài toán trước khi code}: Định nghĩa rõ ràng 4 category (unripe/export/overripe/defective) và tiêu chí phân loại ngay từ đầu giúp tránh được nhiều rắc rối về sau.
    
    \item \textbf{Test sớm, test thường xuyên}: Unit test và integration test giúp phát hiện lỗi sớm, đặc biệt quan trọng khi làm việc với model ML có tính không xác định.
    
    \item \textbf{Cân bằng giữa accuracy và inference speed}: Trong ứng dụng real-time, đôi khi chấp nhận giảm accuracy vài phần trăm để có trải nghiệm mượt mà hơn là lựa chọn hợp lý.
    
    \item \textbf{Document hóa cẩn thận}: Ghi chép rõ các quyết định thiết kế, tham số huấn luyện và kết quả thử nghiệm giúp việc debug và cải tiến về sau dễ dàng hơn nhiều.
\end{enumerate}

\section{Hướng phát triển}

\subsection{Ngắn hạn (1-3 tháng)}

\begin{enumerate}
    \item \textbf{Train detector riêng}: Thu thập và gán nhãn bounding box cho dataset chuối thực tế (đa dạng nền, góc chụp, ánh sáng) để thay thế COCO detector, giảm tình trạng miss detection.
    
    \item \textbf{Fine-tune với dữ liệu ``cầm tay''}: Bổ sung ảnh chuối được cầm trên tay vào training set để cải thiện accuracy trong kịch bản sử dụng phổ biến nhất.
    
    \item \textbf{Tối ưu hóa cho CPU}: Áp dụng các kỹ thuật như model pruning, quantization (INT8) để tăng FPS trên các thiết bị không có GPU.
    
    \item \textbf{Hoàn thiện export TFLite}: Test kỹ model TFLite trên Android thực tế, phát triển app Android với CameraX + TFLite Interpreter.
\end{enumerate}

\subsection{Trung hạn (3-6 tháng)}

\begin{enumerate}
    \item \textbf{Mở rộng sang các loại trái cây khác}: Áp dụng kiến trúc tương tự cho táo, cam, xoài --- các loại quả có đặc điểm thị giác tương tự (thay đổi màu khi chín).
    
    \item \textbf{Instance segmentation}: Nâng cấp từ bounding box sang mask segmentation để tách riêng từng quả trong nải, xử lý tốt hơn trường hợp chuối chồng lấp.
    
    \item \textbf{Tích hợp edge device}: Deploy hệ thống lên Raspberry Pi hoặc NVIDIA Jetson Nano cho các ứng dụng nhúng trong nhà kho, dây chuyền phân loại.
    
    \item \textbf{Cloud API}: Xây dựng REST API cho phép tích hợp với các hệ thống quản lý kho hàng, ERP.
\end{enumerate}

\subsection{Dài hạn (6-12 tháng)}

\begin{enumerate}
    \item \textbf{Dự đoán shelf-life}: Từ các đặc trưng hiện tại, mở rộng sang bài toán dự đoán ``thời gian còn lại trước khi hỏng'' --- một bài toán regression thách thức hơn nhưng có giá trị thực tiễn cao.
    
    \item \textbf{Active learning pipeline}: Xây dựng hệ thống tự động thu thập mẫu khó (hard samples) từ production và đề xuất annotation, liên tục cải thiện model.
    
    \item \textbf{Multi-modal fusion}: Kết hợp thông tin từ camera RGB với các cảm biến khác (NIR spectroscopy, electronic nose) để đánh giá chất lượng toàn diện hơn.
    
    \item \textbf{Explainable AI}: Nghiên cứu và tích hợp các kỹ thuật như Grad-CAM, SHAP để giải thích \textit{tại sao} model đưa ra quyết định --- đặc biệt quan trọng trong các ứng dụng yêu cầu accountability.
\end{enumerate}

\section{Lời kết}

Đề tài ``Ứng dụng thị giác máy tính trong phân loại chất lượng Chuối'' không chỉ là một bài tập về ứng dụng công nghệ mà còn là hành trình khám phá ranh giới giữa nhận thức của con người và khả năng học tập của máy. Từ những đường viền màu HSV đơn giản đến các feature map phức tạp trong mạng nơ-ron tích chập, từ việc ``dạy'' máy phân biệt xanh-vàng-nâu đến việc nó tự học được các pattern tinh vi mà đôi khi chính chúng ta cũng không thể diễn đạt bằng lời --- mỗi bước đều mở ra những suy ngẫm về bản chất của ``hiểu biết'' và ``nhận thức''.

\textbf{Những bài học vượt ra ngoài kỹ thuật:}

Quá trình thực hiện đề tài đã mang lại những bài học sâu sắc vượt ra ngoài khuôn khổ kỹ thuật:

\begin{itemize}
    \item \textbf{Constraints là cơ hội}: Việc thiếu bounding box annotation đã thúc đẩy chúng tôi suy nghĩ sáng tạo về kiến trúc 2-stage, một giải pháp cuối cùng mang lại nhiều lợi ích không ngờ.
    
    \item \textbf{Transfer learning như một triết lý}: Không ai bắt đầu từ số không. Việc kế thừa tri thức từ các mô hình pretrained không chỉ là chiến lược kỹ thuật mà còn phản ánh cách con người học hỏi --- chúng ta luôn xây dựng trên nền tảng của những người đi trước.
    
    \item \textbf{Sự khác biệt giữa ``pattern matching'' và ``hiểu biết''}: Mô hình của chúng tôi ``nhận ra'' chuối chín với accuracy 98.75\%, nhưng nó không ``hiểu'' khái niệm ``chín'' theo cách con người hiểu. Nhận thức điều này giúp đặt kỳ vọng đúng đắn và sử dụng công cụ AI một cách có trách nhiệm.
\end{itemize}

\textbf{Hướng về tương lai:}

Hy vọng rằng những kết quả và kinh nghiệm từ đề tài này sẽ đóng góp một phần nhỏ vào việc ứng dụng trí tuệ nhân tạo trong nông nghiệp Việt Nam, hướng tới một nền nông nghiệp thông minh, hiệu quả và bền vững hơn.

\vspace{0.5cm}
\begin{flushright}
\textit{``The question is not whether intelligent machines can have any emotions, but whether machines can be intelligent without any emotions.''}

--- Marvin Minsky
\end{flushright}

\vspace{1cm}

\begin{center}
\rule{5cm}{0.5pt}
\end{center}

\clearpage
